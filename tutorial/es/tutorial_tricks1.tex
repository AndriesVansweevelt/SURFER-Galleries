\documentclass[es]{../../common/SurferDesc}%%%%%%%%%%%%%%%%%%%%%%%%%%%%%%%%%%%%%%%%%%%%%%%%%%%%%%%%%%%%%%%%%%%%%%%
%
% The document starts here:
%
\begin{document}
\footnotesize
% Einfache Singularit�ten 
%

\begin{surferPage}
  \begin{surferTitle}Trucos de Experto I\end{surferTitle}
   \begin{surferText}

Ya aprendiste mucho sobre el SURFER. Ahora te mostraremos algunos trucos para expertos:\\
\vspace{0.3cm}
Si multiplic{\'a}s las ecuaciones de dos superficies, {\'e}stas se mostrar{\'a}n juntas. Si ahora le rest{\'a}s un valor peque{\~n}o a ese producto, la intersecci{\'o}n de las curvas se vuelve m{\'a}s suave. Es decir, las dos superficies se funden.\\
\vspace{0.3cm}
En la f{\'o}rmula, al restarle el par{\'a}metro $a$, {\'e}ste estar{\'a} inicializado en cero. Pero al incrementarlo, vas a ver c{\'o}mo las superficies se van fundiendo entre s{\'i}.

   \end{surferText}
\end{surferPage}

 



\end{document}
%
% end of the document.
%
%%%%%%%%%%%%%%%%%%%%%%%%%%%%%%%%%%%%%%%%%%%%%%%%%%%%%%%%%%%%%%%%%%%%%%%
