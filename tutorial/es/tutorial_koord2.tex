\begin{surferPage}[Ejes Cartesianos II]{El Sistema de Coordenadas II}
La ecuaci{\'o}n
\[xy=0\]
muestra lo que parece un sistema de coordenadas en dos dimensiones; pero no lo es. Es como si estuvieras "viendo desde arriba" dos planos perpendiculares; es por eso que no ves la tercera dimensi{\'o}n. El eje $z$ est{\'a} como escondido. \\
\vspace{0.3cm}
De $x=0$ se obtiene el plano-$yz$ , de $y=0$ el plano-$xz$.
Ambas son ecuaciones con el cero en la derecha. Si las multiplicamos, se mostrar{\'a}n juntas; ya que el producto de dos factores es cero si alguno de ellos lo es. Esto significa que las dos superficies fueron {\it agregadas}. \\
De esta manera podr{\'i}amos agregar un n{\'u}mero arbitrario de superficies. Por otro lado, ser{\'i}a cada vez m{\'a}s y m{\'a}s dif{\'i}cil su c{\'a}lculo.
\end{surferPage}
