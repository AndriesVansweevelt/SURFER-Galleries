\documentclass[es]{./../../common/SurferDesc}%%%%%%%%%%%%%%%%%%%%%%%%%%%%%%%%%%%%%%%%%%%%%%%%%%%%%%%%%%%%%%%%%%%%%%%
%
% The document starts here:
%
\begin{document}
\footnotesize
% Einfache Singularit�ten 
%
\begin{surferPage}
  \begin{surferTitle}The sphere\end{surferTitle}
   \begin{surferText}
   
The equation of the circle only contains the variables $x$ and $y$. We are still in the two-dimensional space.
The equation of the circle is:
\[x^2+y^2=r^2.\]
If you turn the surface, you can see a tube. This is because there is no restriction along the $z$ axis. If you replace the variable x with z, you still get a tube.\\
Now add the missing term as a square (for example $z^2$ to the circle equation). 
This leads to a sphere:
\[x^2+y^2+z^2=r^2,\]
in SURFER-format
\[0=x^2+y^2+z^2-a^2.\]
What happens, if you turn the sphere?

     \end{surferText}
\end{surferPage}


\end{document}
%
% end of the document.
%
%%%%%%%%%%%%%%%%%%%%%%%%%%%%%%%%%%%%%%%%%%%%%%%%%%%%%%%%%%%%%%%%%%%%%%%
