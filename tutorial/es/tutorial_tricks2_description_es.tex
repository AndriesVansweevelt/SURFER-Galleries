\documentclass[es]{../../common/SurferDesc}%%%%%%%%%%%%%%%%%%%%%%%%%%%%%%%%%%%%%%%%%%%%%%%%%%%%%%%%%%%%%%%%%%%%%%%
%
% The document starts here:
%
\begin{document}
\footnotesize
% Einfache Singularit�ten 
%

 
 \begin{surferPage}
  \begin{surferTitle}Trucos de Experto II\end{surferTitle}
   \begin{surferText}

La curva que resulta de la intersecci{\'o}n de los dos cilindros del ejemplo anterior es una l{\'i}nea infinitamente delgada. Una l{\'i}nea infinitamente delgada es una l{\'i}nea dibujada con un l{\'a}piz de grosor 0. Usando la f{\'o}rmula
\[ f^2+g^2-a=0\]
vas a poder calcular la curva resultante de la intersecci{\'o}n de las superficies $f$ y $g$. Dado que la suma $f^2+g^2$  es igual a $0$ s{\'o}lo si los dos t{\'e}rminos son $0$, si $a=0$ vas a tener la curva que resulta de la intersecci{\'o}n de las superficies dadas por $f=0$ y $g=0$.
 El par{\'a}metro $a$ distorsiona la ecuaci{\'o}n: le agrega espesor al l{\'a}piz con el cual dibujamos la curva que conforma la intersecci{\'o}n de las superficies. Si modific{\'a}s $a$, vas a hacer que dicha curva de intersecci{\'o}n sea m{\'a}s visible
\newline \newline
!`Felicitaciones! !`Ya sos un experto en el SURFER! !`Divertite creando nuevas formas!
 


     \end{surferText}
\end{surferPage}

%%% Local Variables: 
%%% mode: latex
%%% TeX-master: "jDM08_expl"
%%% End: 



\end{document}
%
% end of the document.
%
%%%%%%%%%%%%%%%%%%%%%%%%%%%%%%%%%%%%%%%%%%%%%%%%%%%%%%%%%%%%%%%%%%%%%%%
