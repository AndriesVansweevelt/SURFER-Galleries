\documentclass[es]{../../common/SurferDesc}%%%%%%%%%%%%%%%%%%%%%%%%%%%%%%%%%%%%%%%%%%%%%%%%%%%%%%%%%%%%%%%%%%%%%%%
%
% The document starts here:
%
\begin{document}
\footnotesize
% Einfache Singularitten 
%

%%% Local Variables: 
%%% mode: latex
%%% TeX-master: "jDM08_expl"
%%% End: 

\begin{surferPage}
  \begin{surferTitle}La Circunferencia\end{surferTitle}
   \begin{surferText}
   
Una circunferencia se describe mediante la ecuaci{\'o}n:
\[x^2+y^2=r^2\]
Donde $r$ es el radio de la circunferencia. 
Para escribir esta f{\'o}rmula en el SURFER, tenés que hacerlo as{\'i}:
\[x^2+y^2-a^2=0\]

Donde el par{\'a}metro $a$ denota el radio, y pod{\'e}s modificarlo usando la barra deslizante. Además, pod{\'e}s agregar un segundo par{\'a}metro $b$ para {\it aplastar} la circunferencia y convertirlo en una elipse, por ejemplo con:
\[bx^2+y^2-a^2=0\] donde $b=0.5$.


     \end{surferText}
\end{surferPage}


\end{document}
%
% end of the document.
%
%%%%%%%%%%%%%%%%%%%%%%%%%%%%%%%%%%%%%%%%%%%%%%%%%%%%%%%%%%%%%%%%%%%%%%%
