\documentclass[es]{../../common/SurferDesc}%%%%%%%%%%%%%%%%%%%%%%%%%%%%%%%%%%%%%%%%%%%%%%%%%%%%%%%%%%%%%%%%%%%%%%%
%
% The document starts here:
%
\begin{document}
\footnotesize
% Einfache Singularitäten 
%

\begin{surferPage}
  \begin{surferTitle}El Plano\end{surferTitle}
   \begin{surferText}

La ecuaci\'on del plano es \[x=0\] No contiene informaci\'on sobre $y$ ni sobre $z$. Esto significa que ni la variable $y$ ni la variable $z$ est\'an restringidas, pueden tomar valores cualesquiera. La ecuaci\'on $x=0$ se verifica para todos los puntos con coordenada $x$ igual a cero y coordenadas $y$ y $z$ arbitrarias. Este es el plano $yz$.
\newline \newline
¿Pero por qu\'e un plano infinitamente grande se ve como un c\'irculo? La respuesta est\'a escondida en el programa: \'este muestra la porci\'on de superficie dentro de una esfera invisible y entonces s\'olo podemos ver lo que est\'e dentro de la misma. Si el programa no hiciera esto, el plano no entrar\'ia en la pantalla.

     \end{surferText}
\end{surferPage}

%%%%%end
%%% Local Variables: 
%%% mode: latex
%%% TeX-master: "jDM08_expl"
%%% End: 



\end{document}
%
% end of the document.
%
%%%%%%%%%%%%%%%%%%%%%%%%%%%%%%%%%%%%%%%%%%%%%%%%%%%%%%%%%%%%%%%%%%%%%%%
