\begin{surferPage}{Tips voor experts II}
De snijkromme van de twee cilinders uit het vorige voorbeeld is een oneindig dunne lijn. Stel je hier even voor dat zo'n lijn wordt getekend met een potlood van dikte nul. Met de formule
\[ f^2+g^2-a=0\]
kunnen we de snijkromme berekenen van twee oppervlakken $f$ en $g$. Omdat de som $f^2+g^2$ van twee kwadraten slechts nul kan zijn als de beide termen gelijk zijn aan nul, vind je voor $a=0$ de snijkromme, gegeven door $f=0$ \'en $g=0$.
 De parameter $a$ vervormt de vergelijking: ze maakt het potlood waarmee de snijkromme wordt getekend een beetje dikker. Als je $a$ laat vari\"eren maak je de snijkromme meer zichtbaar.
\newline \newline
Proficiat, je bent nu een echte SURFER-expert. Veel plezier met het bedenken en cre\"eren van nieuwe oppervlakken!
\end{surferPage}
