\begin{surferPage}{Citrus}
In het tekstvak onderaan kan je een vergelijking ingeven. SURFER berekent dan het bijhorende oppervlak en zorgt voor een interactieve afbeelding. Het rechterlid van de vergelijking is steeds gelijk aan nul. 
\\
De vergelijking kan de volgende elementen bevatten:
\newline
Variabelen:
\[x, y, z,\ldots \]
Co\"effici\"enten in de vorm van getallen of parameters:
\[1, 2, 3, \dots a, b, c, \dots \]
Rekenkundige bewerkingen:
\[+,  - , \cdot \quad \textnormal{en} \]
Exponenten:
\[ ^2, ^3, ^n .\]
Elk punt van de ruimte wordt voorgesteld door drie getallen: waardes voor elk van de veranderlijken  $x$, $y$ en $z$. Als je de veranderlijken in de vergelijking vervangt door de waarden van een punt en nul bekomt, zal het punt afgebeeld worden. Met een methode die {\it ray tracing} genoemd wordt vindt SURFER al deze punten, die we de nulpunten van de vergelijking noemen. Alle nulpunten samen vormen dan het oppervlak.
\end{surferPage}
