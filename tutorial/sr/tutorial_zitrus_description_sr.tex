\documentclass[sr]{./../../common/SurferDesc}%%%%%%%%%%%%%%%%%%%%%%%%%%%%%%%%%%%%%%%%%%%%%%%%%%%%%%%%%%%%%%%%%%%%%%%
%
% The document starts here:
%
\begin{document}
\footnotesize
% Einfache Singularitäten 
%
\begin{surferPage}
  \begin{surferTitle}Лимун
  \end{surferTitle}   %%% Zitrus

У доње поље можете да унесете формулу. SURFER  након тога израчунава одговарајућу површ и приказује је. Формула са десне стране мора увек да буде једнака нули.
\\
Једначина може да садржи следеће елементе:
\newline
Променљиве:
\[x, y, z, \]
Коефицијенте у облику бројева или параметара:
\[1, 2, 3, \dots a, b, c, \dots, \]
Аритметичке операције:
\[+,  - , \cdot \quad \textnormal{and} \]
Експоненте:
\[ ^2, ^3, ^n .\]
Свака тачка координатног система је представљена помоћу вредности три променљиве $x$, $y$ и $z$. Ако је у одређеној тачки вредност једначине нула, та тачка се приказује. Помоћу методе зване \textit{ray tracing} програм проналази све такве тачке; оне се називају нуле једначине и све заједно образују површ.

  
  \begin{surferText}
     \end{surferText}
\end{surferPage}

%%%%%%%%%%%%%%%%%%%%%%%%%%%%%%%%

%%% Local Variables: 
%%% mode: latex
%%% TeX-master: "jDM08_expl"
%%% End: 



\end{document}
%
% end of the document.
%
%%%%%%%%%%%%%%%%%%%%%%%%%%%%%%%%%%%%%%%%%%%%%%%%%%%%%%%%%%%%%%%%%%%%%%%
