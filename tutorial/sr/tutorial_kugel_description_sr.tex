\documentclass[en]{./../../common/SurferDesc}%%%%%%%%%%%%%%%%%%%%%%%%%%%%%%%%%%%%%%%%%%%%%%%%%%%%%%%%%%%%%%%%%%%%%%%
%
% The document starts here:
%
\begin{document}
\footnotesize
% Einfache Singularitäten 
%
\begin{surferPage}
  \begin{surferTitle}Сфера\end{surferTitle}
   \begin{surferText}
   
Једначина кружнице садржи само променљиве $x$ и $y$. Још увек смо у дводимензионом простору.
Једначина кружнице је:
\[x^2+y^2=r^2.\]
Ако окренете површ, видећете цев јер нема ограничења дуж $z$ осе. Ако замените променљиву x променљивом  z, још увек добијате цев.\\
Сада додајте члан који недостаје али као квадрат (на пример $z^2$ једначини кружнице). 
Тиме добијамо једначину сфере:
\[x^2+y^2+z^2=r^2,\]
или у формату програма SURFER
\[0=x^2+y^2+z^2-a^2.\]
Шта ће се десити ако окренете сферу?

     \end{surferText}
\end{surferPage}


\end{document}
%
% end of the document.
%
%%%%%%%%%%%%%%%%%%%%%%%%%%%%%%%%%%%%%%%%%%%%%%%%%%%%%%%%%%%%%%%%%%%%%%%
