\documentclass[sr]{./../../common/SurferDesc}%%%%%%%%%%%%%%%%%%%%%%%%%%%%%%%%%%%%%%%%%%%%%%%%%%%%%%%%%%%%%%%%%%%%%%%
%
% The document starts here:
%
\begin{document}
\footnotesize
% Einfache Singularitäten 
%

 
 \begin{surferPage}
  \begin{surferTitle}Савети стручњака 2\end{surferTitle}
   \begin{surferText}

Пресечна крива два ваљка у претходном примеру је бесконачно танка права. Бесконачно танка права је права нацртана оловком дебљине 0. Користећи формулу
\[ f^2+g^2-a=0\]
можемо да израчунамо пресечну криву површи $f$ и $g$. Како је збир квадрата $f^2+g^2$ нула само ако су оба сабирка нула, добијамо да је за $a=0$ крива пресека дата са $f=0$ and $g=0$.
Параметар $a$ мења једначину: додаје дебљину оловки којом се црта пресечна крива. Ако мењате вредност $a$ постићи ћете да се пресечна крива боље види.
\newline \newline
Честитамо, постали сте стручњак за SURFER. Уживајте правећи и смишљајући нове слике!
 


     \end{surferText}
\end{surferPage}

%%% Local Variables: 
%%% mode: latex
%%% TeX-master: "jDM08_expl"
%%% End: 



\end{document}
%
% end of the document.
%
%%%%%%%%%%%%%%%%%%%%%%%%%%%%%%%%%%%%%%%%%%%%%%%%%%%%%%%%%%%%%%%%%%%%%%%
