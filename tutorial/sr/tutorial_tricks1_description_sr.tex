\documentclass[sr]{./../../common/SurferDesc}%%%%%%%%%%%%%%%%%%%%%%%%%%%%%%%%%%%%%%%%%%%%%%%%%%%%%%%%%%%%%%%%%%%%%%%
%
% The document starts here:
%
\begin{document}
\footnotesize
% Einfache Singularitäten 
%

\begin{surferPage}
  \begin{surferTitle}Савети стручњака 1\end{surferTitle}
   \begin{surferText}

Доста сте научили о програму. Сада вам дајемо неколико савета:\\
\vspace{0.3cm}
Ако помножите једначине две површи, оне ће бити додате једна другој. Ако одузмете малу вредност од овог производа, пресечна крива ће постати глаткија. То значи да се површи стапају.\\
\vspace{0.3cm}
На почетку је вредност параметра $a$ који се одузима 0 али ако га повећавате, посматраћете како се површи стапају.

   \end{surferText}
\end{surferPage}

 



\end{document}
%
% end of the document.
%
%%%%%%%%%%%%%%%%%%%%%%%%%%%%%%%%%%%%%%%%%%%%%%%%%%%%%%%%%%%%%%%%%%%%%%%
