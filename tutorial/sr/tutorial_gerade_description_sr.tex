\documentclass[en]{./../../common/SurferDesc}%%%%%%%%%%%%%%%%%%%%%%%%%%%%%%%%%%%%%%%%%%%%%%%%%%%%%%%%%%%%%%%%%%%%%%%
%
% The document starts here:
%
\begin{document}
\footnotesize
% Einfache Singularit�ten 
%
 
\begin{surferPage}
  \begin{surferTitle}The Line\end{surferTitle}
   \begin{surferText}
   
This surface shows an image we know from the school: a coordinate system, given by  $xy=0$, and a line. \\Its formula is:
\[y=a\cdot x + b\]
\[ \Rightarrow \quad a\cdot x +b -y=0.\]
The parameter $a$ is the slope of the line and the parameter $b$ is the distance between the line and the origin of the coordinate system.
\newline \newline
The values fo $a$ und $b$ are not fixed. On the right side two sliders appeared to change the values of the parameters. If you change the values you can see how slope and distance to the origin increase or decrease for different values.

     \end{surferText}
\end{surferPage}
%%%End:

\end{document}
%
% end of the document.
%
%%%%%%%%%%%%%%%%%%%%%%%%%%%%%%%%%%%%%%%%%%%%%%%%%%%%%%%%%%%%%%%%%%%%%%%
