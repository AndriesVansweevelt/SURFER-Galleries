\documentclass[en]{./../../common/SurferDesc}%%%%%%%%%%%%%%%%%%%%%%%%%%%%%%%%%%%%%%%%%%%%%%%%%%%%%%%%%%%%%%%%%%%%%%%
%
% The document starts here:
%
\begin{document}
\footnotesize
% Einfache Singularitäten 
%
 
\begin{surferPage}
  \begin{surferTitle}Права\end{surferTitle}
   \begin{surferText}
   
Ова површ приказује слику коју знамо из школе: координатни систем, дефинисан са $xy=0$, и права. \\Једначина праве је:
\[y=a\cdot x + b\]
\[ \Rightarrow \quad a\cdot x +b -y=0.\]
Параметар  $a$ означава нагиб праве а параметар  $b$ растојање праве од координатног почетка.
\newline \newline
Вредности  $a$ и $b$ нису фиксиране. Клизачи са десне стране омогућавају да се вредности $a$ и $b$ мењају. Ако им мењате вредности видећете како нагиб и растојање од координатног почетка расту или опадају за разне вредности.

     \end{surferText}
\end{surferPage}
%%%End:

\end{document}
%
% end of the document.
%
%%%%%%%%%%%%%%%%%%%%%%%%%%%%%%%%%%%%%%%%%%%%%%%%%%%%%%%%%%%%%%%%%%%%%%%
