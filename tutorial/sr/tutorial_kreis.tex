\documentclass[sr]{./../../common/SurferDesc}%%%%%%%%%%%%%%%%%%%%%%%%%%%%%%%%%%%%%%%%%%%%%%%%%%%%%%%%%%%%%%%%%%%%%%%
%
% The document starts here:
%
\begin{document}
\footnotesize
% Einfache Singularitäten 
%

%%% Local Variables: 
%%% mode: latex
%%% TeX-master: "jDM08_expl"
%%% End: 

\begin{surferPage}
  \begin{surferTitle}Кружница\end{surferTitle}
   \begin{surferText}
   
Једначина која описује кружницу 
\[x^2+y^2=r^2.\]
Да бисмо је унели у програм SURFER морамо да је напишемо у облику
\[x^2+y^2-a^2=0.\]
Параметар $a$ означава полупречник $r$ кружнице и може да се мења помоћу клизача. Сада можете да додате други параметар $b$ којим се кружница ''пригњечи'' и настаје елипса. На пример:
\[bx^2+y^2-a^2=0\] где је $b=0.5$.


     \end{surferText}
\end{surferPage}


\end{document}
%
% end of the document.
%
%%%%%%%%%%%%%%%%%%%%%%%%%%%%%%%%%%%%%%%%%%%%%%%%%%%%%%%%%%%%%%%%%%%%%%%
