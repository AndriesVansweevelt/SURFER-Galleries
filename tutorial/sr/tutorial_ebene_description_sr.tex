\documentclass[en]{./../../common/SurferDesc}%%%%%%%%%%%%%%%%%%%%%%%%%%%%%%%%%%%%%%%%%%%%%%%%%%%%%%%%%%%%%%%%%%%%%%%
%
% The document starts here:
%
\begin{document}
\footnotesize
% Einfache Singularit�ten 
%

\begin{surferPage}
  \begin{surferTitle}The Plane\end{surferTitle}
   \begin{surferText}

The equation for the plane is \[x=0.\] It contains no information on $y$ and $z$. This means that $y$ and $z$ are not constrained, they can take any arbitrary value.
The equation $x=0$ holds for all points with an $x$ value of zero and arbitrary $y$ and $z$ values. This is the $yz$-plane.
\newline \newline
But why does an infinitely big plane look like a filled circle? The answer is hidden in the programme. It displays the surface always inside an invisible sphere and we can only see what is inside this sphere. Else wise it would not fit onto our screen

     \end{surferText}
\end{surferPage}

%%%%%end
%%% Local Variables: 
%%% mode: latex
%%% TeX-master: "jDM08_expl"
%%% End: 



\end{document}
%
% end of the document.
%
%%%%%%%%%%%%%%%%%%%%%%%%%%%%%%%%%%%%%%%%%%%%%%%%%%%%%%%%%%%%%%%%%%%%%%%
