\documentclass[sr]{./../../common/SurferDesc}%%%%%%%%%%%%%%%%%%%%%%%%%%%%%%%%%%%%%%%%%%%%%%%%%%%%%%%%%%%%%%%%%%%%%%%
%
% The document starts here:
%
\begin{document}
\footnotesize
% Einfache Singularitäten 
%

\begin{surferPage}
  \begin{surferTitle}Коцка\end{surferTitle}
   \begin{surferText}
   
Ако повећавате вредност експонената у једначини сфере десиће се нешто изненађујуће. Сфера постаје коцка са заобљеним теменима.\\
\vspace{0.3cm}
Али ово не важи за све експоненте. Једначина коцке је:
\[x^n+y^n+z^n=b\]
где број $n$ мора да буде паран.\\
\vspace{0.3cm}
Шта се дешава ако изаберете непаран број? Како се коцка мења ако даље повећавате вредност парног експонента?
     \end{surferText}
\end{surferPage}


\end{document}
%
% end of the document.
%
%%%%%%%%%%%%%%%%%%%%%%%%%%%%%%%%%%%%%%%%%%%%%%%%%%%%%%%%%%%%%%%%%%%%%%%
