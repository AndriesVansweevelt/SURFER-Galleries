\documentclass[en]{./../../common/SurferDesc}%%%%%%%%%%%%%%%%%%%%%%%%%%%%%%%%%%%%%%%%%%%%%%%%%%%%%%%%%%%%%%%%%%%%%%%
%
% The document starts here:
%
\begin{document}
\footnotesize
% Einfache Singularit�ten 
%

%%%%%%%%%%%%%%%%%%%%%%%%%%%%%%%%

\begin{surferPage}
  \begin{surferTitle}The Coordinate System I \end{surferTitle}
   \begin{surferText}
   
Here you see a coordinate system made out of tubes that are placed around the actual axes. The axes itself are infinitely thin, so we use these tubes to be able to see them.\\
The coordinate system describes our three-dimensional space. If you move left or right, you move along the $x$-axis, if you move up and down you move along the $y$-axis and if you move forwards or backwards, then you move along the $z$-axis. The directions are not fixed since you can turn the coordinate system.\\
\vspace{0.3cm}
To turn it use your finger and drag the coordinate system. To visualise the rotation a small representation of the coordinate system is displayed while turning.
     \end{surferText}
\end{surferPage}


\end{document}
%
% end of the document.
%
%%%%%%%%%%%%%%%%%%%%%%%%%%%%%%%%%%%%%%%%%%%%%%%%%%%%%%%%%%%%%%%%%%%%%%%
