\documentclass[sr]{./../../common/SurferDesc}%%%%%%%%%%%%%%%%%%%%%%%%%%%%%%%%%%%%%%%%%%%%%%%%%%%%%%%%%%%%%%%%%%%%%%%
%
% The document starts here:
%
\begin{document}
\footnotesize
% Einfache Singularitäten 
%

%%%%%%%%%%%%%%%%%%%%%%%%%%%%%%%%

\begin{surferPage}
  \begin{surferTitle}Координатни систем 1 \end{surferTitle}
   \begin{surferText}
   
Овде можете видети координатни систем направљен од цеви које су стављене преко стварних оса. Осе су саме по себи бесконачно танке, тако да користимо цеви да бисмо их видели.\\
Координатни систем описује наш тродимензиони простор. Ако се померате лево или десно, померате се по $x$-оси, ако се померате горе или доле, померате се по $y$-оси а ако се померате напред или назад, померате се по $z$-оси. Ови правци оса нису фиксирани, с обзиром да се цео координатни систем може окретати.\\
\vspace{0.3cm}
Да бисте га окренули, користите ваш прст и повуците координатни систем. Да бисте видели ротацију, мали координатни систем се појављује док траје окретање.
     \end{surferText}
\end{surferPage}


\end{document}
%
% end of the document.
%
%%%%%%%%%%%%%%%%%%%%%%%%%%%%%%%%%%%%%%%%%%%%%%%%%%%%%%%%%%%%%%%%%%%%%%%
