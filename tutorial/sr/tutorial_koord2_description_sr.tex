\documentclass[sr]{./../../common/SurferDesc}%%%%%%%%%%%%%%%%%%%%%%%%%%%%%%%%%%%%%%%%%%%%%%%%%%%%%%%%%%%%%%%%%%%%%%%
%
% The document starts here:
%
\begin{document}
\footnotesize
% Einfache Singularitäten 
%

\begin{surferPage}
  \begin{surferTitle}Координатни систем 2\end{surferTitle}
   \begin{surferText}
   
Координатни систем дефинисан са  
\[xy=0\]
није прави координатни систем. Чине га две равни које се секу. Ви посматрате равни {\it одозго}. На овај начин је трећа димензија тј. $z$-оса сакривена. \\
\vspace{0.3cm}
Ако је $x=0$ добијамо $yz$-раван, а ако је $y=0$ добијамо $xz$-раван.
Обе једначине имају са десне стране ''једнако нули''. Ако их помножите, обе ће бити истовремено приказане јер је производ нула ако му је један од чинилаца нула. Ово значи да су површи  {\it додате} једна другој. \\
На овај начин можемо да додајемо произвољан број површи. Међутим, у исто време постаје све теже и теже да их израчунавамо.
\end{surferText}
\end{surferPage}




\end{document}
%
% end of the document.
%
%%%%%%%%%%%%%%%%%%%%%%%%%%%%%%%%%%%%%%%%%%%%%%%%%%%%%%%%%%%%%%%%%%%%%%%
