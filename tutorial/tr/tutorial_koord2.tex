\begin{surferPage}[Koordinatlar II]{Koordinat Sistemi II}
Burada gördüğünüz ve
\[xy=0\]
olarak verilen iki boyutta koordinat sistemi aslında bir koordinat sistemi değil; birbirini kesen iki düzlem söz konusu. Düzlemlere {\it yukarıdan} bakıyorsunuz. Bu şekilde üçüncü boyut, yani $z$ ekseni gizlenmiş. \\
\vspace{0.3cm}
$x=0$ eşitliğiyle  $yz$ düzlemini,   $y=0$ ile de  $xz$ düzlemini elde ederiz.
İki denklemin de sağ tarafında ''eşittir sıfır'' ifadesi var. İkisini çarparak yazdığımızda
düzlemler birlikte gösteriliyor çünkü bir çarpım ancak ve ancak çarpanlardan biri sıfırsa sıfırdır. 
Bu da yüzeylerin {\it eklenmesi} anlamına geliyor. \\
Bu yolla istediğimiz sayıda yüzeyi birbirine ekleyebiliriz. Öte yandan bunları hesaplamak gittikçe zorlaşır.
\end{surferPage}
