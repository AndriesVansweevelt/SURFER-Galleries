\begin{surferPage}{Limon}
Aşağıdaki alana bir formül yazabilirsiniz. SURFER karşılık gelen yüzeyi hesaplayacak ve gösterecektir. Formülün sağ tarafı hep sıfıra eşit.
\\
Denklem şu unsurları içerebilir:
\newline
Değişkenler:
\[x, y, z, \]
Sayı ya da parametre olabilecek katsayılar:
\[1, 2, 3, \dots a, b, c, \dots, \]
Aritmetik işlemler:
\[+,  - , \cdot \quad \textnormal{ve} \]
Üsler:
\[ ^2, ^3, ^n .\]
Koordinat sisteminin her bir noktası, $x$, $y$ ve $z$ değişkenlerinin değerleriyle temsil edilir. Bir noktayı anlatan üç değeri değişkenler yerine koyduğumuzda sıfır çıkıyorsa, o nokta ekranda gösterilecektir. Yazılım, bu tür tüm noktaları \textit{ışın izleme} adıyla anılan bir yöntem aracılığıyla bulur; bu noktalara denklemin sıfırları denir. Tüm bu noktalar birleşerek böylece yüzeyi oluşturur.
\end{surferPage}
