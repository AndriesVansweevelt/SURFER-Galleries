\begin{surferPage}{Uzman Sırları II}
Bir önceki örnekteki iki silindirin kesişim eğrisi sonsuz ince bir çizgidir. 
Sonsuz ince bir çizgi kalınlığı 0 olan bir kalemle çizilmiş bir çizgidir. 
\[ f^2+g^2-a=0\]
formülünü kullanarak $f$ ve $g$ yüzeylerinin kesişim eğrilerini hesaplayabiliriz.
İki karenin toplamı ancak ve ancak iki terimin de 0 olması durumunda 0 olduğundan
bu formülle  $a=0$ iken $f=0$ ve $g=0$ ile verilen eğriyi elde ederiz.
Burada $a$ parametresi eşitliği bozar: kesişim eğrisinin çizildiği kalemin kalınlığını arttırır.
$a$'yı değiştirdikçe kesişim eğrisini gitgide daha görünür kılabilirsiniz.
Eğer eğriyi ince fakat güzel görmek istiyorsanız $a$'yı çok azaltmanız gerekir.
Bunun için formülde $a$ yerine örneğin $0.001\cdot a$ yazmayı deneyin. 
\newline \newline
Eh, tebrikler! Artık siz de bir  SURFER uzmanısınız. Yeni biçimler yaratarak ve keşfederek hoşça vakit geçirin!
\end{surferPage}
