\begin{surferPage}{Zitrus}
아래의 창에 수식을 입력하면 SURFER는 그 식에 해당하는 곡면을 보여줍니다. 수식의 우변은 항상 0 이어야 합니다.
\\
식들은 다음과 같은 요소들을 포함할 수 있습니다:
\newline
변수:
\[x, y, z, \]
변수나 수식으로 된 계수:
\[1, 2, 3, \dots a, b, c, \dots, \]
연산자:
\[+,  - , \cdot \quad \textnormal{그리고} \]
거듭제곱:
\[ ^2, ^3, ^n .\]
좌표계의 모든 점들은 $x$, $y$ 그리고 $z$ 순서쌍에 의해 정해집니다. 입력한 수식을 만족하는 순서쌍은 오른쪽에 화면에 나타나게 됩니다. \textit{ray tracing} 이라는 방법을 통해 이 프로그램은 식을 만족하는 모든 순서쌍을 찾아냅니다. 우리는 그 순서쌍을 식의 해라고 부릅니다. 이 점들을 모두 모으면 곡면이 됩니다.
\end{surferPage}
