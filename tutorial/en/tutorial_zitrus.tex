\documentclass[en]{./../../common/SurferDesc}%%%%%%%%%%%%%%%%%%%%%%%%%%%%%%%%%%%%%%%%%%%%%%%%%%%%%%%%%%%%%%%%%%%%%%%
%
% The document starts here:
%
\begin{document}
\footnotesize
% Einfache Singularit�ten 
%
\begin{surferPage}
  \begin{surferTitle}Zitrus
  \end{surferTitle}   %%% Zitrus

In the field below you can enter a formula. SURFER then computes the associated surface and displays it. The formula on the right always equals zero.
\\
The equation can contain the following elements:
\newline
Variables:
\[x, y, z, \]
Coefficients in form of numbers or parameters:
\[1, 2, 3, \dots a, b, c, \dots, \]
Arithmetic operations:
\[+,  - , \cdot \quad \textnormal{and} \]
Exponents:
\[ ^2, ^3, ^n .\]
Every point of the coordinate system is represented by a value for each of the three variables  $x$, $y$ and $z$. If the term replacing the variables with the values of this point equals zero, then the point will be displayed. Through a method called \textit{ray tracing} the programme finds all such points; they are called zeroes of the equation. All these points together then form the surface.

  
  \begin{surferText}
     \end{surferText}
\end{surferPage}

%%%%%%%%%%%%%%%%%%%%%%%%%%%%%%%%

%%% Local Variables: 
%%% mode: latex
%%% TeX-master: "jDM08_expl"
%%% End: 



\end{document}
%
% end of the document.
%
%%%%%%%%%%%%%%%%%%%%%%%%%%%%%%%%%%%%%%%%%%%%%%%%%%%%%%%%%%%%%%%%%%%%%%%
