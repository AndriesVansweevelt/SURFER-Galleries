\documentclass[no]{./../../common/SurferDesc}%%%%%%%%%%%%%%%%%%%%%%%%%%%%%%%%%%%%%%%%%%%%%%%%%%%%%%%%%%%%%%%%%%%%%%%
%
% The document starts here:
%
\begin{document}
\footnotesize
% Einfache Singularit�ten 
%
\begin{surferPage}
  \begin{surferTitle}Sitrus \end{surferTitle}   %%% Zitrus
I feltet nedenfor kan du skrive inn en ligning. SURFER beregner da den tilh�rende flaten og tegner den. 
Til h�yre for likhetstegnet m� det alltid st� null. 

\\
Ligningen kan inneholde f�lgende elementer:
\newline
Variable:
\[x, y, z, \]
Koeffisienter som tall eller parametere:
\[1, 2, 3, \dots a, b, c, \dots, \]
Regneoperasjoner:
\[+,  - , \cdot \quad \textnormal{and} \]
Eksponenter:
\[ ^2, ^3, ^n .\]

Hvert punkt p� koordinatsystemet er bestemt av en verdi for hver av de tre variablene $x$, $y$ og $z$. 
Dersom ligningen blir null n�r vi setter inn disse verdiene for $x$, $y$ og $z$, vil punktet bli fargelagt. 
Ved hjelp av en metode kalt \textit{ray tracing} finner programmet alle punkter som gj�r at ligningen blir null. 
Alle de fargelagte punktene utgj�r til sammen flaten vi ser p� skjermen. 
  
  \begin{surferText}
     \end{surferText}
\end{surferPage}

%%%%%%%%%%%%%%%%%%%%%%%%%%%%%%%%

%%% Local Variables: 
%%% mode: latex
%%% TeX-master: "jDM08_expl"
%%% End: 



\end{document}
%
% end of the document.
%
%%%%%%%%%%%%%%%%%%%%%%%%%%%%%%%%%%%%%%%%%%%%%%%%%%%%%%%%%%%%%%%%%%%%%%%
