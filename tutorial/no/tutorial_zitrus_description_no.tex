\documentclass[no]{./../../common/SurferDesc}%%%%%%%%%%%%%%%%%%%%%%%%%%%%%%%%%%%%%%%%%%%%%%%%%%%%%%%%%%%%%%%%%%%%%%%
%
% The document starts here:
%
\begin{document}
\footnotesize
% Einfache Singularitäten 
%
\begin{surferPage}
  \begin{surferTitle}Sitrus \end{surferTitle}   %%% Zitrus
I feltet nedenfor kan du skrive inn en ligning. SURFER beregner da den tilhørende flaten og tegner den. 
Til høyre for likhetstegnet må det alltid stå null. 

Ligningen kan inneholde følgende elementer:
\newline
Variable:
\[x, y, z, \]
Koeffisienter som tall eller parametere:
\[1, 2, 3, \dots a, b, c, \dots, \]
Regneoperasjoner:
\[+,  - , \cdot \quad \textnormal{and} \]
Eksponenter:
\[ ^2, ^3, ^n .\]

Hvert punkt på koordinatsystemet er bestemt av en verdi for hver av de tre variablene $x$, $y$ og $z$. 
Dersom ligningen blir null når vi setter inn disse verdiene for $x$, $y$ og $z$, vil punktet bli fargelagt. 
Ved hjelp av en metode kalt \textit{ray tracing} finner programmet alle punkter som gjør at ligningen blir null. 
Alle de fargelagte punktene utgjør til sammen flaten vi ser på skjermen. 
  
  \begin{surferText}
     \end{surferText}
\end{surferPage}

%%%%%%%%%%%%%%%%%%%%%%%%%%%%%%%%

%%% Local Variables: 
%%% mode: latex
%%% TeX-master: "jDM08_expl"
%%% End: 



\end{document}
%
% end of the document.
%
%%%%%%%%%%%%%%%%%%%%%%%%%%%%%%%%%%%%%%%%%%%%%%%%%%%%%%%%%%%%%%%%%%%%%%%
