\documentclass[no]{./../../common/SurferDesc}%%%%%%%%%%%%%%%%%%%%%%%%%%%%%%%%%%%%%%%%%%%%%%%%%%%%%%%%%%%%%%%%%%%%%%%
%
% The document starts here:
%
\begin{document}
\footnotesize
% Einfache Singularit�ten 
%

 
 \begin{surferPage}
  \begin{surferTitle}Eksperttips II\end{surferTitle}
   \begin{surferText}

Skj�ringskurven til de to sylindrene i forrige eksempel, er en uendelig tynn linje. Det er en linje som er tegnet med en blyant av tykkelse 0. Ved � bruke formelen 
\[ f^2+g^2-a=0\]

kan vi beregne skj�ringskurvene til de to flatene $f$ og $g$. Siden summen $f^2+g^2$ av to kvadrater bare er 
$0$ hvis begge leddene er $0$, f�r du for $a=0$ skj�ringskurven gitt av $f=0$ and $g=0$. 
Parameteren $a$ fordreier ligningen: Den gir tykkelse til blyanten som skj�ringskurven er 
tegnet med. Hvis du endrer $a$, kan du gj�re skj�ringskurven mer synlig. 
 
\newline \newline
Gratulerer, n� er du ekspert p� SURFER. Lykke til med � pr�ve ut og skape nye former! 


     \end{surferText}
\end{surferPage}

%%% Local Variables: 
%%% mode: latex
%%% TeX-master: "jDM08_expl"
%%% End: 



\end{document}
%
% end of the document.
%
%%%%%%%%%%%%%%%%%%%%%%%%%%%%%%%%%%%%%%%%%%%%%%%%%%%%%%%%%%%%%%%%%%%%%%%
