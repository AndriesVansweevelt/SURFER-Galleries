\documentclass[no]{./../../common/SurferDesc}%%%%%%%%%%%%%%%%%%%%%%%%%%%%%%%%%%%%%%%%%%%%%%%%%%%%%%%%%%%%%%%%%%%%%%%
%
% The document starts here:
%
\begin{document}
\footnotesize
% Einfache Singularitäten 
%
\begin{surferPage}
  \begin{surferTitle}Kula\end{surferTitle}
   \begin{surferText}
   
I ligningen til sirkelen finner vi bare variablene $x$ og $y$. Vi er altså ennå i det todimensjonale rommet. 
Ligningen til sirkelen er:
\[x^2+y^2=r^2.\]

Hvis du dreier på flaten, kan du se et rør. Røret kommer til syne fordi det ikke er 
noen begrensning på forflytning langs $z$-aksen. Hvis du bytter ut variabelen x med z, får du fortsatt 
et rør. \\

Nå kan du legge til et tredje ledd i z-retning til sirkelligningen, for eksempel $z^2$. 

Da får vi en kule:
\[x^2+y^2+z^2=r^2,\]
i SURFER-format
\[0=x^2+y^2+z^2-a^2.\]
Hva skjer hvis du dreier kula?

     \end{surferText}
\end{surferPage}


\end{document}
%
% end of the document.
%
%%%%%%%%%%%%%%%%%%%%%%%%%%%%%%%%%%%%%%%%%%%%%%%%%%%%%%%%%%%%%%%%%%%%%%%
