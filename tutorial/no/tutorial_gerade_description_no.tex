\documentclass[no]{./../../common/SurferDesc}%%%%%%%%%%%%%%%%%%%%%%%%%%%%%%%%%%%%%%%%%%%%%%%%%%%%%%%%%%%%%%%%%%%%%%%
%
% The document starts here:
%
\begin{document}
\footnotesize
% Einfache Singularitäten 
%
 
\begin{surferPage}
  \begin{surferTitle}Linjen\end{surferTitle}
   \begin{surferText}
   
Denne flaten viser et bilde vi kjenner fra skolen: et koordinatsystem, gitt ved $xy=0$ og en linje. \\Formelen til linjen er:
\[y=a\cdot x + b\]
\[ \Rightarrow \quad a\cdot x +b -y=0.\]
Parameteren $a$ er stigningstallet til linjen, og parameteren $b$ er avstanden mellom linjen og skjæringspunktet mellom koordinatsystemets akser (origo). 
\newline

Verdiene for $a$ og $b$ er ikke faste. På høyre side finner du to glidebrytere du kan bevege på 
for å endre verdiene til parametrene. Her kan du se hvordan stigningen til linjen eller avstanden 
til origo øker eller minker for ulike verdier av $a$ og $b$.
     \end{surferText}
\end{surferPage}
%%%End:

\end{document}
%
% end of the document.
%
%%%%%%%%%%%%%%%%%%%%%%%%%%%%%%%%%%%%%%%%%%%%%%%%%%%%%%%%%%%%%%%%%%%%%%%
