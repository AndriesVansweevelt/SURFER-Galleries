\documentclass[no]{./../../common/SurferDesc}%
\usepackage{csquotes}
%%%%%%%%%%%%%%%%%%%%%%%%%%%%%%%%%%%%%%%%%%%%%%%%%%%%%%%%%%%%%%%%%%%%%%
%
% The document starts here:
%
\begin{document}
\footnotesize
% Einfache Singularitäten 
%
\begin{surferPage}
  \begin{surferTitle}Parabelen\end{surferTitle}
   \begin{surferText}
   
Parabelen er gitt ved ligningen \[y=x^2.\]
For å tilpasse ligningen til formelfeltet, må vi skrive formelen på en slik måte at den er lik null:
\[x^2-y=0\]
Det lille \enquote{taket} i formelfeltet symboliserer at tallet etterpå er eksponenten. Altså er
\[ x  \,\hat{\ } \, 2 =x^2.\]
Vi legger til to parametre:
\[a \cdot x^2-y+b=0.\]
Med parameteren $a$ kan vi komprimere eller utvide parabelen, mens vi med parameteren $b$ kan endre parabelens avstand til origo. 
\newline
Sett parametrene $a$ og $b$ i SURFER-formelen og endre dem. Hvordan må du endre formelen for å snu parabelen opp ned?
\end{surferText}
\end{surferPage}


\end{document}
%
% end of the document.
%
%%%%%%%%%%%%%%%%%%%%%%%%%%%%%%%%%%%%%%%%%%%%%%%%%%%%%%%%%%%%%%%%%%%%%%%
