\documentclass[no]{./../../common/SurferDesc}%%%%%%%%%%%%%%%%%%%%%%%%%%%%%%%%%%%%%%%%%%%%%%%%%%%%%%%%%%%%%%%%%%%%%%%
%
% The document starts here:
%
\begin{document}
\footnotesize
% Einfache Singularit�ten 
%

\begin{surferPage}
  \begin{surferTitle}Planet\end{surferTitle}
   \begin{surferText}
Ligningen for et plan er \[x=0.\]

Den inneholder ingen informasjon om $y$ og $z$, og de kan dermed ta en hvilken som helst verdi. Ligningen \[x=0.\] gjelder
 for alle punkter hvor $x$ har verdien $0$, uansett verdi p� $y$ og $z$. Dette er $yz$-planet.
\newline \newline
Men hvorfor ser et uendelig stort plan ut som en fylt sirkel? Svaret er gjemt i programmet. Det viser 
alltid flaten inne i en usynlig kule, og vi kan bare se det som er inne i denne kula. Ellers 
ville ikke flaten passe inn i skjermbildet. 
     \end{surferText}
\end{surferPage}

%%%%%end
%%% Local Variables: 
%%% mode: latex
%%% TeX-master: "jDM08_expl"
%%% End: 



\end{document}
%
% end of the document.
%
%%%%%%%%%%%%%%%%%%%%%%%%%%%%%%%%%%%%%%%%%%%%%%%%%%%%%%%%%%%%%%%%%%%%%%%
