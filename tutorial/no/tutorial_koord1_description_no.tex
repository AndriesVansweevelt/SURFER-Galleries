\documentclass[no]{./../../common/SurferDesc}%%%%%%%%%%%%%%%%%%%%%%%%%%%%%%%%%%%%%%%%%%%%%%%%%%%%%%%%%%%%%%%%%%%%%%%
%
% The document starts here:
%
\begin{document}
\footnotesize
% Einfache Singularit�ten 
%

%%%%%%%%%%%%%%%%%%%%%%%%%%%%%%%%

\begin{surferPage}
  \begin{surferTitle}Koordinatsystemet I \end{surferTitle}
   \begin{surferText}
   
Her ser du et koordinatsystem laget av r�r som er lagt rundt de egentlige aksene. Selve aksene er uendelig tynne, s� vi bruker disse r�rene for � se dem.\\
Koordinatsystemet beskriver det tredimensjonale rommet. Hvis du beveger deg til venstre eller h�yre, flytter du deg langs $x$-aksen. Beveger du deg opp eller ned, flytter du deg langs $y$-aksen, 
og er bevegelsen rettet framover eller bakover, er det z-aksen du flytter deg langs. Retningene er ikke faste, siden du kan dreie koordinatsystemet rundt. \\

\vspace{0.3cm}
For � dreie p� koordinatsystemet, bruker du fingeren til � dra det et stykke bortover. Underveis vises en liten variant av koordinatsystemet for � visualisere rotasjonen. 
     \end{surferText}
\end{surferPage}


\end{document}
%
% end of the document.
%
%%%%%%%%%%%%%%%%%%%%%%%%%%%%%%%%%%%%%%%%%%%%%%%%%%%%%%%%%%%%%%%%%%%%%%%
