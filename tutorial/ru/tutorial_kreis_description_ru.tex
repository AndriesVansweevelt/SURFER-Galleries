\documentclass[ru]{./../../common/SurferDesc}%%%%%%%%%%%%%%%%%%%%%%%%%%%%%%%%%%%%%%%%%%%%%%%%%%%%%%%%%%%%%%%%%%%%%%%
%
% The document starts here:
%
\begin{document}
\footnotesize
% Einfache Singularitäten 
%

%%% Local Variables: 
%%% mode: latex
%%% TeX-master: "jDM08_expl"
%%% End: 

\begin{surferPage}
  \begin{surferTitle}Окружность\end{surferTitle}
   \begin{surferText}
   
Уравнение, описывающее окружность, следующее:
\[x^2+y^2=r^2.\]
Переформулируем это уравнение для того, чтобы его можно было ввести в поле формул программы SURFER:
\[x^2+y^2-a^2=0.\]
Параметр $a$ – это радиус окружности, его можно изменять при помощи бегунка. Вы можете ввести дополнительный параметр $b$, чтобы расплющить окружность, т.е. сделать из него эллипс, например, так:
\[bx^2+y^2-a^2=0\] при $b=0.5$.

     \end{surferText}
\end{surferPage}


\end{document}
%
% end of the document.
%
%%%%%%%%%%%%%%%%%%%%%%%%%%%%%%%%%%%%%%%%%%%%%%%%%%%%%%%%%%%%%%%%%%%%%%%
