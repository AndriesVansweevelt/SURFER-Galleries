\documentclass[ru]{./../../common/SurferDesc}%%%%%%%%%%%%%%%%%%%%%%%%%%%%%%%%%%%%%%%%%%%%%%%%%%%%%%%%%%%%%%%%%%%%%%%
%
% The document starts here:
%
\begin{document}
\footnotesize
% Einfache Singularitäten 
%

\begin{surferPage}
  \begin{surferTitle}I. Профессиональные трюки\end{surferTitle}
   \begin{surferText}

Теперь Вы уже хорошо знаете программу. И мы можем перейти к профессиональным трюкам:

если умножить уравнения двух поверхностей, то эти поверхности объединятся. Если из результата вычесть небольшую величину, то «кромка» поверхности будет сглаживаться. Т.е. поверхности растворяются друг в друге.

В формуле мы вычтем параметр $a$. Сейчас у него нулевое значение. Если вы его будете увеличивать, то увидите их слияние.

   \end{surferText}
\end{surferPage}

 



\end{document}
%
% end of the document.
%
%%%%%%%%%%%%%%%%%%%%%%%%%%%%%%%%%%%%%%%%%%%%%%%%%%%%%%%%%%%%%%%%%%%%%%%
