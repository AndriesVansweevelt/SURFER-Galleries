\begin{surferPage}{Das Koordinatensystem II}
Das Koordinatenkreuz, das aus der Formel 
\[xy=0\]
entsteht ist, eigentlich gar kein Kreuz, sondern es besteht aus zwei sich schneidenden Ebenen. Man sieht sozusagen {\it von oben} auf die Ebenenkreise. Dadurch wird die dritte Dimension, also die $z$-Achse ausgeblendet.  \\
\vspace{0,3cm}
F�r $x=0$ ergibt sich die $yz$-Ebene und f�r $y=0$ die $xz$-Ebene. 
Beide Gleichungen haben auf der rechten Seite eine Null stehen. Wenn man sie nun miteinander multipliziert, werden sie unabh�ngig voneinander gezeichnet - bei einem Produkt kann ein Faktor, wenn er Null ist, die gesamte Gleichung auf Null setzen - das heisst, die Fl�chen werden vereinigt, man sagt auch {\it addiert}. \\
Auf diese Weise lassen sich theoretisch beliebig viele Fl�chen hinzumultiplizieren. Nur wird es f�r den SURFER immer schwieriger sie zu berechnen.
\end{surferPage}
