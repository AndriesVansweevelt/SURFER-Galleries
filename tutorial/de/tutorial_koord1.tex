\begin{surferPage}[Koordinaten I]{Das Koordinatensystem I}
Hier sieht man ein {\it Koordinatensystem}, das aus Röhren besteht, die um die eigentlichen Achsen gelegt werden. Die Achsen sind eigentlich unendlich dünn, darum werden sie durch die Röhren veranschaulicht. \\
Das Koordinatensystem beschreibt den dreidimensionalen Raum. Wenn man nach links oder rechts läuft, bewegt man sich entlang der $x$-Achse, läuft man nach oben und unten, dann bewegt man sich entlang der $y$-Achse und wenn man nach vorne oder hinten läuft, dann bewegt man sich entlang der $z$-Achse. Die Richtungen sind hier aber nicht festgelegt, da man das Koordinatensystem drehen kann. \newline \newline
Tippen Sie dazu mit dem Finger auf den Bildschirm und fahren auf ihr ein Stück entlang. Schon dreht sich das Koordinatenkreuz. Zur Veranschaulichung der Drehung wird im unteren Bereich auch immer ein kleines Koordinatenkreuz eingezeichnet.
\end{surferPage}
