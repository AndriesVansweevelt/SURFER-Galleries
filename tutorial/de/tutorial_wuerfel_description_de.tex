\documentclass[de]{./../../common/SurferDesc}%%%%%%%%%%%%%%%%%%%%%%%%%%%%%%%%%%%%%%%%%%%%%%%%%%%%%%%%%%%%%%%%%%%%%%%
%
% The document starts here:
%
\begin{document}
\footnotesize
% Einfache Singularit�ten 
%

\begin{surferPage}
  \begin{surferTitle}Ein W�rfel\end{surferTitle}
   \begin{surferText}
   
Wenn man die Exponenten in der Kugelgleichung erh�ht, passiert etwas �berraschendes. Die Kugel formt sich zu einem W�rfel mit abgerundeten Kanten.\\
\vspace{0.3cm}

Doch das gilt nicht f�r alle Exponenten.
Die Gleichung f�r einen W�rfel lautet:
\[x^n+y^n+z^n=b\]
wobei $n$ immer eine gerade Zahl sein muss.\\
\vspace{0.3cm}
Was passiert, wenn man f�r $n$ eine ungerade Zahl einsetzt?
Wie ver�ndert sich der W�rfel, wenn man f�r $n$ immer gr��ere, gerade Zahlen einsetzt?
     \end{surferText}
\end{surferPage}


\end{document}
%
% end of the document.
%
%%%%%%%%%%%%%%%%%%%%%%%%%%%%%%%%%%%%%%%%%%%%%%%%%%%%%%%%%%%%%%%%%%%%%%%
