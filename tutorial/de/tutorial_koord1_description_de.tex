\documentclass[de]{./../../common/SurferDesc}%%%%%%%%%%%%%%%%%%%%%%%%%%%%%%%%%%%%%%%%%%%%%%%%%%%%%%%%%%%%%%%%%%%%%%%
%
% The document starts here:
%
\begin{document}
\footnotesize
% Einfache Singularit�ten 
%

%%%%%%%%%%%%%%%%%%%%%%%%%%%%%%%%

\begin{surferPage}
  \begin{surferTitle}Das Koordinatensystem I \end{surferTitle}
   \begin{surferText}
   
Hier sieht man ein {\it Koordinatensystem}, das aus R�hren besteht, die um die eigentlichen Achsen gelegt werden. Die Achsen sind eigentlich unendlich d�nn, darum werden sie durch die R�hren veranschaulicht. \\
Das Koordinatensystem beschreibt den dreidimensionalen Raum. Wenn man nach links oder rechts l�uft, bewegt man sich entlang der $x$-Achse, l�uft man nach oben und unten, dann bewegt man sich entlang der $y$-Achse und wenn man nach vorne oder hinten l�uft, dann bewegt man sich entlang der $z$-Achse. Die Richtungen sind hier aber nicht festgelegt, da man das Koordinatensystem drehen kann. \newline \newline
Tippen Sie dazu mit dem Finger auf den Bildschirm und fahren auf ihr ein St�ck entlang. Schon dreht sich das Koordinatenkreuz. Zur Veranschaulichung der Drehung wird im unteren Bereich auch immer ein kleines Koordinatenkreuz eingezeichnet.

     \end{surferText}
\end{surferPage}


\end{document}
%
% end of the document.
%
%%%%%%%%%%%%%%%%%%%%%%%%%%%%%%%%%%%%%%%%%%%%%%%%%%%%%%%%%%%%%%%%%%%%%%%
