\documentclass[de]{./../../common/SurferDesc}%%%%%%%%%%%%%%%%%%%%%%%%%%%%%%%%%%%%%%%%%%%%%%%%%%%%%%%%%%%%%%%%%%%%%%%
%
% The document starts here:
%
\begin{document}
\footnotesize
% Einfache Singularit�ten 
%

%%% Local Variables: 
%%% mode: latex
%%% TeX-master: "jDM08_expl"
%%% End: 

\begin{surferPage}
  \begin{surferTitle}Der Kreis\end{surferTitle}
   \begin{surferText}
   
Die Gleichung, die einen Kreis beschreibt, lautet: 
\[x^2+y^2=r^2.\]
Man formt diese Gleichung um, damit man sie in das Formelfeld des SURFERS schreiben kann
\[x^2+y^2-a^2=0\]
Der Parameter $a$ ist hierbei der Radius $r$ des Kreises und kann mit dem Regler ver�ndert werden. Sie k�nnen nun einen zweiten Parameter $b$ hinzunehmen, um den Kreis zu quetschen, also eine Ellipse daraus zu machen, z.B. mit:
\[bx^2+y^2-a^2=0\] mit $b=0.5$.


     \end{surferText}
\end{surferPage}


\end{document}
%
% end of the document.
%
%%%%%%%%%%%%%%%%%%%%%%%%%%%%%%%%%%%%%%%%%%%%%%%%%%%%%%%%%%%%%%%%%%%%%%%
