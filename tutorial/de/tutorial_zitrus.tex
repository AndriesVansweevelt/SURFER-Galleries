\begin{surferPage}{Zitrus}
In der unteren Zeile k�nnen Sie selbst eine Formel eingeben. Der SURFER rechnet sofort die dazugeh�rige Fl�che aus und zeigt sie an. Dabei ist rechts vom Gleichheitszeichen immer eine Null. \\
Eine Gleichung kann aus folgenden Elementen bestehen:
\newline
Variablen:
\[x, y, z, \]
Koeffizienten als Zahlen oder Parameter:
\[1, 2, 3, \dots a, b, c, \dots, \]
Rechenoperationen:
\[+,  - , \cdot \quad \textnormal{und} \]
Exponenten:
\[ ^2, ^3, ^n .\]
Jeder Punkt im Koordinatensystem ist durch einen Wert f�r die drei Variablen $x$, $y$ und $z$ bestimmt. Ergibt der Term beim Einsetzen dieser Werte f\"ur einen Punkt auf der rechten Seite des Gleichheitszeichens eine Null, so wird er eingef�rbt. Das Programm bestimmt durch ein \textit{ray tracing} genanntes Verfahren alle solchen Punkte des Koordinatensystems; man nennt sie die Nullstellen der Gleichung. Alle eingef\"arbten Punkte zusammen erzeugen dann die Fl\"ache.
\end{surferPage}
