\documentclass[de]{./../../common/SurferDesc}%%%%%%%%%%%%%%%%%%%%%%%%%%%%%%%%%%%%%%%%%%%%%%%%%%%%%%%%%%%%%%%%%%%%%%%
%
% The document starts here:
%
\begin{document}
\footnotesize
% Einfache Singularit�ten 
%
\begin{surferPage}
  \begin{surferTitle}Die Kugel\end{surferTitle}
   \begin{surferText}
   
In der Gleichung des Kreises sind nur die Variablen $x$ und $y$ enthalten. Wir sind also immer noch im zweidimensionalen Raum. 
Die Gleichung des Kreises lautet:
\[x^2+y^2=r^2.\]
Drehen Sie die Fl�che nun so zur Seite, dass man eine R�hre sieht. Diese R�hre entsteht dadurch, da es entlang der $z$-Achse keine Beschr�nkung gibt. Ersetzt man nun die Variable $x$ durch $z$, entsteht wieder eine R�hre.\\
Addieren Sie nun die fehlende Variable als Quadrat (z.B. $z^2$ bei der Kreisgleichung). 
Dadurch entsteht eine Kugel:
\[x^2+y^2+z^2=r^2,\]
in SURFER-Schreibweise
\[0=x^2+y^2+z^2-a^2.\]
Was passiert, wenn man die Kugel dreht?

     \end{surferText}
\end{surferPage}


\end{document}
%
% end of the document.
%
%%%%%%%%%%%%%%%%%%%%%%%%%%%%%%%%%%%%%%%%%%%%%%%%%%%%%%%%%%%%%%%%%%%%%%%
