\begin{surferPage}{Dullo}
Unieke fenomenen in de natuur\\
\smallskip
\[(x^2+ y^2+ z^2)^2	= x^2+ y^2\]

\singlespacing
Wiskunde is nauw verbonden met andere wetenschappen zoals fysica of chemie, en met technologische ontwikkelingen. Ze zorgt voor krachtige instrumenten om de wereld rondom ons te begrijpen. 
\singlespacing
Vele verschijnselen die zich voordoen als we de natuur bestuderen, geven aanleiding tot modellen waar singulariteiten in voorkomen.
\singlespacing
Dit is bijvoorbeeld het geval bij de voortplanting van geluidsgolven die geproduceerd worden door het applaus van een enthousiast publiek in een voetbalstadion. Dit verschijnsel neemt de vorm aan van het Dullo-oppervlak. Dit oppervlak heeft duidelijk een singulier punt in het midden, en daarom zorgt de scheidsrechter ervoor dat hij zich niet op dit punt bevindt als een goal gevierd wordt. Het lawaai zou zijn gehoor kunnen beschadigen!
\end{surferPage}
