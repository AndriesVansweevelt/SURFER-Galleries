\begin{surferPage}{Teek}
De vergelijking, een ondubbelzinnige naam \\
\smallskip
\[x^2 + y^2	= z^3	(1 - z) \]


\singlespacing
Alle figuren in deze galerij hebben namen. Welke naam zou jij ze geven? Dezelfde of een andere?\\
\vspace{0.3cm}
Bestaat er een manier om vormen te benoemen die nooit voor verwarring kan zorgen? De wiskunde heeft hiervoor een oplossing gevonden: door ze namen te geven aan de hand van hun vergelijking. Die bepaalt immers alle punten, krommingen, gaten, keerpunten en plooien van het oppervlak. Je moet enkel weten hoe je deze eigenschappen vindt in de vergelijking en hoe je ze moet tekenen.\\
\vspace{0.3cm}
Vergelijkingen worden over heel de wereld op dezelfde manier geschreven en ge\"interpreteerd. Dit is zo omdat de wiskunde een universele taal is, net als bijvoorbeeldde muzieknotatie. 
\end{surferPage}
