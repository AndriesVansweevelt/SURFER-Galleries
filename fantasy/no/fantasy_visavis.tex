\begin{surferPage}{Vis \`a vis}
Spiss eller glatt – venn eller fiende\\
\smallskip
\[x^2	- x^3+ y^2+ y^4+ z^3- z^4	=  0\]

\vspace{0.3cm}
Singulære punkter, eller singulariteter, kjenner man lett igjen på formen. Dette er punkter hvor flaten ikke er glatt og myk, men for eksempel har en spiss eller en fold.\\
\vspace{0.3cm}
Spissen på venstre side av flaten Vis \`a vis er en singularitet, mens den glatte helningen på høyre side er det ikke. Singulariteter er interessante fordi bare små endringer i ligningen kan endre utseendet deres på en overraskende måte.\\

\vspace{0.3cm}
Visste du at det finnes mennesker som jobber spesielt med å studere disse punktene? Sorte hull og Big Bang er singulariteter i kosmologiske modelligninger som beskriver universet vårt. Og se på fingerspissene dine: Singularitetene på fingeravtrykkene våre identifiserer oss!
\end{surferPage}
