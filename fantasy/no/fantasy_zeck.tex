\documentclass[no]{./../../common/SurferDesc}%%%%%%%%%%%%%%%%%%%%%%%%%%%%%%%%%%%%%%%%%%%%%%%%%%%%%%%%%%%%%%%%%%%%%%%
%
% The document starts here:
%
\begin{document}
\footnotesize
% Einfache Singularitäten 

 
%%%%%%%%%%%%%%%%%%%%%%%%%%%%%%%%

\begin{surferPage}
  \begin{surferTitle}Tikk\end{surferTitle}  \\
Ligningen, med et tydelig navn \\
\smallskip
\[x^2 + y^2	= z^3	(1 - z) \]


\singlespacing
Alle figurer i dette galleriet har navn. Hva ville du kalt dem? Hvilke navn ville en annen person sett seg ut?\\
\vspace{0.3cm}
Kan vi finne en måte å navngi former på som ikke skaper forvirring? Matematikere har funnet en løsning: ved å gi dem navn etter ligningen deres. Den bestemmer alle punktene, kurvene, hullene, foldene og spissene til formen. Du trenger bare å vite hvordan du finner disse formene i ligningen og hvordan de framstår i tegningen av flaten.\\
\vspace{0.3cm}
Ligninger er skrevet og tolket på samme måte verden over, fordi det matematiske språket er universelt, akkurat som noter i musikken. 

  \begin{surferText}
     \end{surferText}
\end{surferPage}



\end{document}
%
% end of the document.
%
%%%%%%%%%%%%%%%%%%%%%%%%%%%%%%%%%%%%%%%%%%%%%%%%%%%%%%%%%%%%%%%%%%%%%%%
