\documentclass[no]{./../../common/SurferDesc}%%%%%%%%%%%%%%%%%%%%%%%%%%%%%%%%%%%%%%%%%%%%%%%%%%%%%%%%%%%%%%%%%%%%%%%
%
% The document starts here:
%
\begin{document}
\footnotesize
% Einfache Singularitäten 

\begin{surferPage}
  \begin{surferTitle}Ding Dong\end{surferTitle}  \\
Endre figuren ved å endre ligningen\\

\smallskip
\[x^2	+ y^2	+ z^3	= z^2\]

\singlespacing
Ligningen og formen til Ding dong er enkel. Figuren lages ved å dreie den greske bokstaven Alfa rundt sin egen akse. Hvis du ser på den opp ned, ser Ding dong ut som en dråpe vann. Vi kan nesten se hvordan dråpen faller! 
\newline
Hvis du legger til en liten parameter $a$ til ligningen og endrer den gradvis, kan du skape en bildeserie som viser bevegelsen til dråpen, hvor den ender sin ferd og til slutt løser seg opp. Det er som stillbilder av en film:

\[x^2	+ y^2	+ z^3	-z^2+0.1\cdot a=0.\]

Hele veien er dråpen i likevekt, siden tyngdekraften veier opp for overflatespenningen. Men likevekten er ustabil, og dråpen skjelver inntil den bryter sammen. Katastrofeteorien til René Thom viser hvordan små forandringer i parametrene kan føre til øyeblikkelige endringer i likevekten. 


  \begin{surferText}
     \end{surferText}
\end{surferPage}
%%%%%%%%%%%%%%%%%%%%%%%%%%%%%


%%%%%%%%%%%%%%%%%%%%%%%%%%%%


\end{document}
%
% end of the document.
%
%%%%%%%%%%%%%%%%%%%%%%%%%%%%%%%%%%%%%%%%%%%%%%%%%%%%%%%%%%%%%%%%%%%%%%%
