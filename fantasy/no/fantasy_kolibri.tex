\documentclass[no]{./../../common/SurferDesc}%%%%%%%%%%%%%%%%%%%%%%%%%%%%%%%%%%%%%%%%%%%%%%%%%%%%%%%%%%%%%%%%%%%%%%%
%
% The document starts here:
%
\begin{document}
\footnotesize
% Einfache Singularitäten 

%%%%%%%%%%%%%%%%%%%%%%%%%%%%%
\begin{surferPage}
  \begin{surferTitle}Kolibri\end{surferTitle}   \\
Ligningen bestemmer punktene\\
  
  \smallskip
\[z^3+ y^2	z^2	= x^2\]

\singlespacing
Uttrykt ved algebra, er Kolibrien gitt av alle punktene $(x, y, z)$ som passer inn i ligningen
\smallskip
\[ x^2= y^2z^2+z^3.\]
\smallskip
For eksempel er, $(0,0,0),$ $(1,0,1)$ og $(3,-2,-3)$ punkter på Kolibrien, mens $(0,1,1)$ ikke tilhører den.\\
 \singlespacing
Vår tredimensjonale verden har tre retninger: forover og bakover, venstre og høyre, opp og ned. Disse retningene betegnes med $x$, $y$ og $z$. Hvert punkt i rommet kan beskrives som en verdi langs hver av retningene. Vi kaller verdiene for koordinatene $(x,y,z)$ til punktet.\\
\singlespacing
Nå plasserer vi verdiene til alle punkter i rommet inn i ligningen og fargelegger bare de som oppfyller ligningen. Alle de fargelagte punktene utgjør til sammen bildet.\\
\singlespacing


  \begin{surferText}
     \end{surferText}
\end{surferPage}


\end{document}
%
% end of the document.
%
%%%%%%%%%%%%%%%%%%%%%%%%%%%%%%%%%%%%%%%%%%%%%%%%%%%%%%%%%%%%%%%%%%%%%%%
