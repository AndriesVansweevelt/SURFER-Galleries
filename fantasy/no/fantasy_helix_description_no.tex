\documentclass[no]{./../../common/SurferDesc}%%%%%%%%%%%%%%%%%%%%%%%%%%%%%%%%%%%%%%%%%%%%%%%%%%%%%%%%%%%%%%%%%%%%%%%
%
% The document starts here:
%
\begin{document}
\footnotesize
% Einfache Singularitäten 

%%%%%%%%%%%%%%%%%%%%%%%%%%%%%%

\begin{surferPage}
  \begin{surferTitle}Helix\end{surferTitle}   \\
Thinner than a soap film\\
  \smallskip
\[6x^2	= 2x^4	+ y^2	z^2\]

\singlespacing
Soap bubbles are sensitive; they appear to burst by just looking at them. Their surfaces have two sides. Outside is the soap and inside water. If the soap layer becomes too thin - this happens if the bubble gets bigger - the water makes the bubble burst.\\
\vspace{0,3cm}
Algebraic surfaces are much thinner than soap films, they are only made out of point layers. And since we use our imagination to create these points, without mass or density, they do not burst, even if they have peaks and wrinkles as Helix.\\
\vspace{0,3cm}
But, if we want to create a three-dimensional model of the Helix surface, we have to build a sculpture thicker than the real Helix surface. This can be done by reinforcing the surface on one side.

  \begin{surferText}
     \end{surferText}
\end{surferPage}

%%%%%%%%%%%%%%%%%%%%%%%%%%%%%%


 

\end{document}
%
% end of the document.
%
%%%%%%%%%%%%%%%%%%%%%%%%%%%%%%%%%%%%%%%%%%%%%%%%%%%%%%%%%%%%%%%%%%%%%%%
