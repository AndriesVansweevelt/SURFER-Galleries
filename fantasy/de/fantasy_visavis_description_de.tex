\documentclass[de]{./../../common/SurferDesc}%%%%%%%%%%%%%%%%%%%%%%%%%%%%%%%%%%%%%%%%%%%%%%%%%%%%%%%%%%%%%%%%%%%%%%%
%
% The document starts here:
%
\begin{document}
\footnotesize
% Einfache Singularitäten 


\begin{surferPage}
  \begin{surferTitle}Vis \`a Vis\end{surferTitle} 
  \\
Spitz oder glatt - Freunde oder Feinde\\
\smallskip
\[x^2	- x^3+ y^2+ y^4+ z^3- z^4	=  0\]

\vspace{0.3cm}
Die spitzen Punkte, die so genannten {\it Singularit"aten}, erkennt man h�ufig an ihrer Form. Es handelt sich um Punkte, an denen die Fl"ache nicht glatt und weich ist, sondern zum Beispiel eine Spitze oder eine Falte hat.\\
\vspace{0.3cm}
Die Fl"ache Vis \`a Vis zeigt sehr gut, was eine Singularit"at ist: die Spitze auf der linken Seite. Und sie zeigt auch, was sie nicht ist: der ebene H"ugel auf der rechten Seite. Singularit"aten sind unter anderem deshalb interessant, weil sie -
im Gegensatz zu stabilen glatten Punkten - ihr Aussehen schon bei kleinen "Anderungen in der Gleichung "uberraschend "andern k"onnen.\\
\vspace{0.3cm}
Wissen Sie, dass es Menschen gibt, die sich speziell dem Studium dieser Punkte widmen? Die schwarzen L"ocher und der Beginn des Universums, der Big Bang, sind Singularit"aten in den Gleichungen der kosmologischen Modelle. Und betrachten Sie Ihre Fingerspitzen: die Singularit�ten unserer Fingerabdr\"ucke  identifizieren uns.


  \begin{surferText}
     \end{surferText}
\end{surferPage}

\end{document}
%
% end of the document.
%
%%%%%%%%%%%%%%%%%%%%%%%%%%%%%%%%%%%%%%%%%%%%%%%%%%%%%%%%%%%%%%%%%%%%%%%
