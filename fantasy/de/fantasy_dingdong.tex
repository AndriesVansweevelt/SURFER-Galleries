\begin{surferPage}{Ding Dong}
Ver\"andern der Figur durch Verändern der Gleichung \\

\smallskip
\[x^2	+ y^2	+ z^3	= z^2\]

\singlespacing
Die Gleichung und die Form von Ding Dong sind sehr einfach. Die Figur erhält man, indem man die Form des griechischen Buchstaben Alpha um seine eigene Achse dreht. Wenn man es umgekehrt betrachtet, sieht Ding Dong wie ein Wassertropfen aus. Man könnte fast sagen, dass wir zusehen, wie der Tropfen fällt!
\newline
Wenn man einen kleinen Parameter $a$ zur Gleichung hinzufügt, und sie kontinuierlich verändert, können wir eine Serie von Bildern erzeugen, die uns die Entstehung des Tropfens zeigt, wie er sich an seine Endposition annähert und sich schlussendlich ablöst. Es ist wie eine Abfolge von Fotoaufnahmen eines Filmes:
\[x^2	+ y^2	+ z^3	-z^2+0.1\cdot a=0.\]

In jedem Moment ist der Tropfen in der Situation eines Gleichgewichts, in der die Schwerkraft die Oberflächenspannung der Flüssigkeit ausgleicht. Aber das Gleichgewicht des Tropfens ist offensichtlich unstabil und die Tropfen zittern bis sie herunterfallen.
Die Katastrophentheorie des Mathematikers Ren\'e Thom beschäftigt sich damit, wie kleine Änderungen in den Parametern sofortige Änderungen des Gleichgewichts bewirken können.
\end{surferPage}
