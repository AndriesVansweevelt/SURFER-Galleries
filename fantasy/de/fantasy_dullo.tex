\documentclass[de]{./../../common/SurferDesc}%%%%%%%%%%%%%%%%%%%%%%%%%%%%%%%%%%%%%%%%%%%%%%%%%%%%%%%%%%%%%%%%%%%%%%%
%
% The document starts here:
%
\begin{document}
\footnotesize
% Einfache Singularitäten 


\begin{surferPage}
  \begin{surferTitle}Dullo\end{surferTitle}  
  \\
Singul"are Ph"anomene der Natur\\
\smallskip
\[(x^2+ y^2+ z^2)^2	= x^2+ y^2\]

\singlespacing
Die Mathematik ist sehr eng mit anderen Wissenschaften, wie der Physik, Chemie oder Technik verkn"upft. Sie liefert m"achtige Werkzeuge, um die Welt zu verstehen, die uns umgibt.
\singlespacing
Viele Ph"anomene, die wir beim Studium der Natur finden, f"uhren zu Modellen mit Singularit"aten. Um sie zu vermeiden, m�ssen wir wissen wo sich die Singularit"aten genau befinden. 
\singlespacing
Die Schallwellen in einem Fu�ballstadion haben ann�hernd die Form von Dullo. In der Mitte hat Dullo eine Singularit�t. An dieser treffen sich die Schallwellen, wenn ein Tor f�llt. Um seine Ohren zu sch�tzen vermeidet es der Schiedsrichter an diesem Ort zu stehen, wenn ein Tor f�llt.


  \begin{surferText}
     \end{surferText}
\end{surferPage}
%%%%%%%%%%%%%%%%


\end{document}
%
% end of the document.
%
%%%%%%%%%%%%%%%%%%%%%%%%%%%%%%%%%%%%%%%%%%%%%%%%%%%%%%%%%%%%%%%%%%%%%%%
