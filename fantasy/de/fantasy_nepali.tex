\begin{surferPage}{Nepali}
Eine Welt ohne Ende \\

\smallskip
\[(x y - z^3 -1)^2= (1 - x^2	- y^2)^3\]

\singlespacing
Vielleicht erscheint Ihnen eine Fl"ache einfach nur sch"on und Sie haben Lust, sie in eine Kristallkugel mit Schnee zu packen, um sie zu sch"utteln und mir ihr zu spielen. Aber glauben Sie nicht, dass Sie jede Fl"ache ausw"ahlen k"onnen, um Sie in Ihr Wohnzimmer zu stellen! \\
\singlespacing
Es gibt Fl"achen, die sich bis ins Unendliche ausdehnen und, so sch"on sie auch sein m"ogen, niemals werden Sie sie in eine Kristallkugel stecken k"onnen, so gro"s die Kugel auch sein mag. In diesem Fall sagt man, dass die Fl"ache \textit{nicht beschr"ankt} ist und um sie zu zeichnen, muss man einen Teil der Fl"ache verstecken. \\
\singlespacing
Die Eigenschaft beschr"ankt zu sein, kann man nicht einfach empirisch erkennen, nicht einmal mit Hilfe von SURFER. Es ist so, also ob wir herausfinden wollen, ob das Universum beschr"ankt ist oder nicht: weil man die R"ander nicht kennt, kann es welche haben oder nicht.
\end{surferPage}
