\documentclass[de]{./../../common/SurferDesc}%%%%%%%%%%%%%%%%%%%%%%%%%%%%%%%%%%%%%%%%%%%%%%%%%%%%%%%%%%%%%%%%%%%%%%%
%
% The document starts here:
%
\begin{document}
\footnotesize
% Einfache Singularitäten 


\begin{surferPage}
  \begin{surferTitle}S"uss\end{surferTitle}   \\

\smallskip
\[(x^2+ 9/4y^2	+ z^2- 1)^3- x^2z^3	- 9/80y^2z^3	= 0\]

\singlespacing
Liebesbrief
\singlespacing
So kann es nun nicht weitergehn! \\
Das, was besteht, mu"s bleiben. \\
Wenn wir uns wieder wiedersehn, \\
Mu"s irgendetwas geschehn. \\
Was wir dann auf die Spitze treiben.\\ 
Was - was auf einer Spitze tut? \\
Gewi"s nicht Plattit"uden. \\
Denn was auf einer Spitze ruht, \\
Wird nicht so leicht erm"uden. \\
Auf einer Bank im Grunewald \\
Zu zweit im Regen sitzen, \\
Ist bl"od. Mut, M"adchen! Schreibe bald! \\
Dein Fritz! (Remember Spitzen). \\
 {\it Gedicht von Joachim Ringelnatz}
\singlespacing 
Die Leidenschaft einer Liebe verbindet man gew"ohnlich mit der emotionellen Kraft einer ''Singularit"at''. Diese Verbindung taucht auch immer wieder in allen Arten der Kunst auf.
\singlespacing 
\"Andern Sie den letzten Exponenten der Gleichung von einer 3 zu einer 2 und beobachten Sie, was passiert.



  \begin{surferText}
     \end{surferText}
\end{surferPage}

\end{document}
%
% end of the document.
%
%%%%%%%%%%%%%%%%%%%%%%%%%%%%%%%%%%%%%%%%%%%%%%%%%%%%%%%%%%%%%%%%%%%%%%%
