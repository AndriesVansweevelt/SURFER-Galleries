\begin{surferPage}{T"ulle}
Unendlich viele Buchstaben in einem Wort\\
\smallskip
\[y z (x^2	+ y - z)	= 0\]


\singlespacing
Auf die gleiche Weise wie die Impressionisten H"auser und Wiesen mit tausenden Farbpunkten gemalt haben, sind die Fl"achen aus tausenden Punkten zusammengesetzt. Wenn man genau ist, aus unendlich vielen, n"amlich aus allen L"osungen der Gleichung!\\
\singlespacing
Eine Art und Weise sich die Unendlichkeit vorzustellen, ist es einfach zu z"ahlen: $1, 2, 3,$ \dots\\
Es gibt immer eine gr"o"sere Zahl und wir werden es nie schaffen, alle nat"urlichen Zahlen aufzuz"ahlen.\\
\singlespacing
Doch nicht nur die Fl�che besteht aus unendlich vielen Punkten. Auf dem Zahlenstrahl liegen zwischen den Zahlen $0$ und $1$ allein unendlich viele Punkte. Das scheint nicht m�glich? Bedenken Sie dass die Punkte unendlich klein sind. Man zeichnet sie sozusagen mit einem Stift der Dicke Null. Man muss ganz sch�n viele Punkte zeichnen, um die Linie zwischen $0$ und $1$ auszuf�llen, n�mlich unendlich viele. 
\end{surferPage}
