\begin{surferPage}{Helix}
D"unner als eine Seifenblase\\
  \smallskip
\[6x^2	= 2x^4	+ y^2	z^2\]

\singlespacing
Seifenblasen sind so sensibel, dass sie durch blo�es hinsehen zerplatzen k�nnen. Ihre Oberfl�chen haben jedoch zwei Seiten. Au�en ist Seife und innen ist Wasser. Wird die Seifenschicht zu d�nn, wenn die Blase gr��er wird, dann zerrei�t das Wasser die Seifenhaut.\\
\vspace{0,3cm}
Algebraische Fl"achen sind noch viel d"unner als Seifenh"aute, sie sind nur aus einer Punktschicht gemacht. Und da wir nur unsere Vorstellungskraft verwenden, um die Punkte zu erzeugen, ohne Masse oder Dicke, zerbrechen die Fl"achen nicht, auch wenn sie Spitzen und Falten haben, die so stark geformt sind wie bei Helix.\\
\vspace{0,3cm}
Aber, wenn wir ein reales, dreidimensionales Modell der Fl"ache von Helix erstellen wollen, m"ussen wir eine Skulptur erzeugen, die dicker ist als die echte Helix-Fl"ache, indem wir die Fl"ache auf einer Seite verst"arken. 
\end{surferPage}
