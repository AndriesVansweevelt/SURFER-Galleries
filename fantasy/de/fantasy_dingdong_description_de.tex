\documentclass[de]{./../../common/SurferDesc}%%%%%%%%%%%%%%%%%%%%%%%%%%%%%%%%%%%%%%%%%%%%%%%%%%%%%%%%%%%%%%%%%%%%%%%
%
% The document starts here:
%
\begin{document}
\footnotesize
% Einfache Singularitäten 

\begin{surferPage}
  \begin{surferTitle}Ding Dong\end{surferTitle}  \\
Ver"andern der Figur durch Ver"andern der Gleichung \\

\smallskip
\[x^2	+ y^2	+ z^3	= z^2\]

\singlespacing
Die Gleichung und die Form von Ding Dong sind sehr einfach. Die Figur erh"alt man, indem man die Form des griechischen Buchstaben Alpha um seine eigene Achse dreht. Wenn man es umgekehrt betrachtet, sieht Ding Dong wie ein Wassertropfen aus. Man k"onnte fast sagen, dass wir zusehen, wie der Tropfen f"allt!
\newline
Wenn man einen kleinen Parameter $a$ zur Gleichung hinzuf"ugt, und sie kontinuierlich ver"andert, k"onnen wir eine Serie von Bildern erzeugen, die uns die Entstehung des Tropfens zeigt, wie er sich an seine Endposition ann"ahert und sich schlussendlich abl"ost. Es ist wie eine Abfolge von Fotoaufnahmen eines Filmes:
\[x^2	+ y^2	+ z^3	-z^2+0.1\cdot a=0.\]

In jedem Moment ist der Tropfen in der Situation eines Gleichgewichts, in der die Schwerkraft die Oberfl"achenspannung der Fl"ussigkeit ausgleicht. Aber das Gleichgewicht des Tropfens ist offensichtlich unstabil und die Tropfen zittern bis sie herunterfallen.
Die Katastrophentheorie des Mathematikers Ren\'e Thom besch�ftigt sich damit, wie kleine "Anderungen in den Parametern sofortige "Anderungen des Gleichgewichts bewirken k"onnen. 



  \begin{surferText}
     \end{surferText}
\end{surferPage}
%%%%%%%%%%%%%%%%%%%%%%%%%%%%%


%%%%%%%%%%%%%%%%%%%%%%%%%%%%


\end{document}
%
% end of the document.
%
%%%%%%%%%%%%%%%%%%%%%%%%%%%%%%%%%%%%%%%%%%%%%%%%%%%%%%%%%%%%%%%%%%%%%%%
