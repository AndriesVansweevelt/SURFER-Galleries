\documentclass[de]{./../../common/SurferDesc}%%%%%%%%%%%%%%%%%%%%%%%%%%%%%%%%%%%%%%%%%%%%%%%%%%%%%%%%%%%%%%%%%%%%%%%
%
% The document starts here:
%
\begin{document}
\footnotesize
% Einfache Singularitäten 

%%%%%%%%%%%

\begin{surferPage}
  \begin{surferTitle}Himmel und H"olle\end{surferTitle}  \\
Wir erschaffen neue Formen \\
\smallskip
\[x^2	- y^2z^2	= 0\]

\singlespacing
Um neue Formen zu erschaffen, muss man Gleichungen verstehen. Die elementaren Teile sind so genannte  {\it Monome}, algebraische Ausdr"ucke mit Buchstaben und Zahlen.
\singlespacing
Ein Monom beinhaltet folgende Elemente:
Vorzeichen, Koeffizient, Variable, Exponent und Grad.\\
\singlespacing
Zum Beispiel: 
\smallskip
\[2xy^2z = +2x^1y^2z^1.\]
\\
\smallskip
Der  {\it Grad} eines Monoms ist die Summe der Exponenten seiner Variablen: $Grad = 1+2+1=4$.  \\
\singlespacing
Um Gleichungen bilden zu k"onnen, benutzt man Rechenoperationen wie Addieren, Subtrahieren und Multiplizieren. Das sind Operationen, die wir in der Grundschule lernen. Damit lassen sich alle algebraischen Fl�chen darstellen.
\singlespacing
Schaffen Sie es, Formen mit Spitzen oder L"ochern zu erzeugen, nur durch Zusammenz"ahlen und Multiplizieren?



  \begin{surferText}
     \end{surferText}
\end{surferPage}
\end{document}
%
% end of the document.
%
%%%%%%%%%%%%%%%%%%%%%%%%%%%%%%%%%%%%%%%%%%%%%%%%%%%%%%%%%%%%%%%%%%%%%%%
