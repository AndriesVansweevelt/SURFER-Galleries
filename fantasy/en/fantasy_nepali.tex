\documentclass[en]{./../../common/SurferDesc}%%%%%%%%%%%%%%%%%%%%%%%%%%%%%%%%%%%%%%%%%%%%%%%%%%%%%%%%%%%%%%%%%%%%%%%
%
% The document starts here:
%
\begin{document}
\footnotesize
% Einfache Singularitäten 

\begin{surferPage}
  \begin{surferTitle}Nepali\end{surferTitle}  \\
Never ending world \\

\smallskip
\[(x y - z^3 -1)^2= (1 - x^2	- y^2)^3\]

\singlespacing
Maybe you find a surface simply beautiful and want to place it in a crystal ball with snow to play with it. But don't think that you can choose any surface to put it into your living room!
\\
\singlespacing
There are surfaces that extend until infinity and, even if they are extremely pretty, you will never be able to place them into a crystal ball, no matter of its size. In this case we call the surface \textit{not constrained}. To paint such a surface we have to hide parts of it.
\\
\singlespacing
The property to be constrained can not be recognised easily, not even with the help of SURFER. It is as if we want to find out if the Universe is constrained: since we do not know its borders, it might have some or not.
  \begin{surferText}
     \end{surferText}
\end{surferPage}
%%%%%%%%%%%%%%%%%%%%%%%%%%%%%
%%%%%%%%%%%%%%%%%%%%%%%%%%%%%


\end{document}
%
% end of the document.
%
%%%%%%%%%%%%%%%%%%%%%%%%%%%%%%%%%%%%%%%%%%%%%%%%%%%%%%%%%%%%%%%%%%%%%%%
