\documentclass[en]{./../../common/SurferDesc}%%%%%%%%%%%%%%%%%%%%%%%%%%%%%%%%%%%%%%%%%%%%%%%%%%%%%%%%%%%%%%%%%%%%%%%
%
% The document starts here:
%
\begin{document}
\footnotesize
% Einfache Singularitäten 

\begin{surferPage}
  \begin{surferTitle}Ding Dong\end{surferTitle}  \\
Change the figure by changing the equation\\

\smallskip
\[x^2	+ y^2	+ z^3	= z^2\]

\singlespacing
The equation and the form of Ding Dong are simple. The figure is obtained by turning the Greek letter Alpha around its axis. If you look at it upside down, Ding Dong looks like a drop of water. We can watch the drop falling.
\newline
If you add a small parameter $a$ to the equation and change it continuously, we can create a series of images that show the emergence of the drop, how it approaches its end position and finally gets separated. It is like still images of a film:

\[x^2	+ y^2	+ z^3	-z^2+0.1\cdot a=0.\]

In every moment the drop is in a situation of balance where gravity compensates the surface tension. But the balance of the drop is not stable and it shivers before falling off. Catastrophe theory by the mathematician Ren\'e Thom studies how small changes in parameters can cause immediate changes in balance. 



  \begin{surferText}
     \end{surferText}
\end{surferPage}
%%%%%%%%%%%%%%%%%%%%%%%%%%%%%


%%%%%%%%%%%%%%%%%%%%%%%%%%%%


\end{document}
%
% end of the document.
%
%%%%%%%%%%%%%%%%%%%%%%%%%%%%%%%%%%%%%%%%%%%%%%%%%%%%%%%%%%%%%%%%%%%%%%%
