\documentclass[en]{./../../common/SurferDesc}%%%%%%%%%%%%%%%%%%%%%%%%%%%%%%%%%%%%%%%%%%%%%%%%%%%%%%%%%%%%%%%%%%%%%%%
%
% The document starts here:
%
\begin{document}
\footnotesize
% Einfache Singularitäten 

 
%%%%%%%%%%%%%%%%%%%%%%%%%%%%%%%%

\begin{surferPage}
  \begin{surferTitle}Tick\end{surferTitle}  \\
The equation, an unambiguous name \\
\smallskip
\[x^2 + y^2	= z^3	(1 - z) \]


\singlespacing
All figures in this gallery have names. How would you have called them? How would another person name them?\\
\vspace{0.3cm}
Can we find a way of naming shapes that never leads to confusion? Mathematics has found a solution: by naming them by their equation. It determines all its points, all curves, holes, wrinkles and peaks. You just have to know how to find these forms inside the formula and how to draw them.\\
\vspace{0.3cm}
Equations are written and interpreted the world over in the same manner, because the language of mathematics is universal, just like musical scores. 

  \begin{surferText}
     \end{surferText}
\end{surferPage}



\end{document}
%
% end of the document.
%
%%%%%%%%%%%%%%%%%%%%%%%%%%%%%%%%%%%%%%%%%%%%%%%%%%%%%%%%%%%%%%%%%%%%%%%
