\documentclass[en]{./../../common/SurferDesc}%%%%%%%%%%%%%%%%%%%%%%%%%%%%%%%%%%%%%%%%%%%%%%%%%%%%%%%%%%%%%%%%%%%%%%%
%
% The document starts here:
%
\begin{document}
\footnotesize
% Einfache Singularitäten 

%%% 1.Tafel
\begin{surferPage}
  \begin{surferTitle} Zitrus\end{surferTitle}  \\ %%% Zitrus
This is not a lemon - the treachery of images\\
\smallskip
\[x^2 + z^2 = y^3 (1 - y)^3\] 


\singlespacing
No doubt when we first set eyes on this image we all think: ``It's a lemon``. But, if it is a lemon, why does it neither have a scent nor a taste? Why doesn't it have pores or spots? Clearly it can't be a lemon! 
\singlespacing
This shape is not a lemon, but a mathematical model of it. It helps us get a better grasp of the properties of the lemon's shape. In geography there is a matching quote by $Alfred\ H.\ S.\ Korzybski$: ''The map is not the territory.'' \\
\singlespacing

Equations allow us to build mathematical models that help us to study the shape of things better. 
\singlespacing
All this is part of the poetry of mathematics: we can generate beautiful surfaces by means of algebraic equations that transport our thoughts to unexpected corners of our mind. 



  \begin{surferText}
     \end{surferText}
\end{surferPage}

\end{document}
%
% end of the document.
%
%%%%%%%%%%%%%%%%%%%%%%%%%%%%%%%%%%%%%%%%%%%%%%%%%%%%%%%%%%%%%%%%%%%%%%%
