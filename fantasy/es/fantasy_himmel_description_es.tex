\documentclass[es]{../../common/SurferDesc}%%%%%%%%%%%%%%%%%%%%%%%%%%%%%%%%%%%%%%%%%%%%%%%%%%%%%%%%%%%%%%%%%%%%%%%
%
% The document starts here:
%
\begin{document}
\footnotesize
% Einfache Singularitäten 

%%%%%%%%%%%

\begin{surferPage}
  \begin{surferTitle}El cielo y el infierno\end{surferTitle}  \\
Creando nuevas formas \\
\smallskip
\[x^2	- y^2z^2	= 0\]

\singlespacing
Para poder crear nuevas formas debemos entender cómo funcionan las ecuaciones. Sus elementos básicos se conocen como  {\it monomios}, que son expresiones algebraicas con letras y números.
\singlespacing
Un monomio debe tener los siguientes elementos:
Signos, coeficientes, variables, exponentes y grado.\\
\singlespacing
Por ejemplo: 
\smallskip
\[2xy^2z = 2x^1y^2z^1.\]
\\
\smallskip
El  {\it grado} de un monomio es la suma de los exponentes de las distintas variables: $grado = 1+2+1=4$.  \\
\singlespacing
Para armar las ecuaciones utilizamos las operaciones básicas de la aritmética suma, resta, multiplicación y división. Todas estas operaciones son las que se aprenden en primaria y es todo lo que se necesita para describir las superficies algebraicas.
\singlespacing
¿Podés crear una superficie con picos y agujeros con solo sumas y multiplicaciones?


  \begin{surferText}
     \end{surferText}
\end{surferPage}
\end{document}
%
% end of the document.
%
%%%%%%%%%%%%%%%%%%%%%%%%%%%%%%%%%%%%%%%%%%%%%%%%%%%%%%%%%%%%%%%%%%%%%%%
