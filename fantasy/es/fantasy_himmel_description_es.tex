\documentclass[es]{./../../common/SurferDesc}%%%%%%%%%%%%%%%%%%%%%%%%%%%%%%%%%%%%%%%%%%%%%%%%%%%%%%%%%%%%%%%%%%%%%%%
%
% The document starts here:
%
\begin{document}
\footnotesize
% Einfache Singularitäten 

%%%%%%%%%%%

\begin{surferPage}
  \begin{surferTitle}Heaven and Hell\end{surferTitle}  \\
We create new forms \\
\smallskip
\[x^2	- y^2z^2	= 0\]

\singlespacing
To create new forms we have to understand how equations work. Their elements are the so called  {\it monomials}, algebraic expressions with letters and numbers.
\singlespacing
A monomial can contain the following elements:
Signs, coefficients, variables, exponents and the degree.\\
\singlespacing
For example: 
\smallskip
\[2xy^2z = +2x^1y^2z^1.\]
\\
\smallskip
The  {\it degree} of a monomial is the sum of the exponents of its variables: $degree = 1+2+1=4$.  \\
\singlespacing
To form equations we use arithmetic operations as addition, subtraction and multiplication. These are operations we know from primary school. They are used to display all algebraic surfaces.
\singlespacing
Can you create shapes with peaks and holes, only by using addition and multiplication?


  \begin{surferText}
     \end{surferText}
\end{surferPage}
\end{document}
%
% end of the document.
%
%%%%%%%%%%%%%%%%%%%%%%%%%%%%%%%%%%%%%%%%%%%%%%%%%%%%%%%%%%%%%%%%%%%%%%%
