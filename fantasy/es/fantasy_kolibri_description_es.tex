\documentclass[es]{./../../common/SurferDesc}%%%%%%%%%%%%%%%%%%%%%%%%%%%%%%%%%%%%%%%%%%%%%%%%%%%%%%%%%%%%%%%%%%%%%%%
%
% The document starts here:
%
\begin{document}
\footnotesize
% Einfache Singularitäten 

%%%%%%%%%%%%%%%%%%%%%%%%%%%%%
\begin{surferPage}
  \begin{surferTitle}Hummingbird\end{surferTitle}   \\
The equation decides the points\\
  
  \smallskip
\[z^3+ y^2	z^2	= x^2\]

\singlespacing
In algebraic terms, Hummingbird is given by all points $(x, y, z)$ that hold the equation
\smallskip
\[ x^2= y^2z^2+z^3.\]
\smallskip
For example, $(0,0,0),$ $(1,0,1)$ and $(3,-2,-3)$ are points of Hummingbird, while $(0,1,1)$ is not part of it.\\
 \singlespacing
 Our three-dimensional world is governed by three directions: ahead and back, left and right, up and down. These directions are identified with $x$, $y$ and $z$. Every point in space can be described by a value for each of its directions. These values are called the coordinates $(x,y,z)$ of this point.\\
\singlespacing
We now place all points in space with their values in the equation and colour only those where the equations is satisfied. All coloured points together then form the image.\\
\singlespacing


  \begin{surferText}
     \end{surferText}
\end{surferPage}


\end{document}
%
% end of the document.
%
%%%%%%%%%%%%%%%%%%%%%%%%%%%%%%%%%%%%%%%%%%%%%%%%%%%%%%%%%%%%%%%%%%%%%%%
