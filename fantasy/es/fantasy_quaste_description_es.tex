\documentclass[es]{SurferDesc}%%%%%%%%%%%%%%%%%%%%%%%%%%%%%%%%%%%%%%%%%%%%%%%%%%%%%%%%%%%%%%%%%%%%%%%
%
% The document starts here:
%
\begin{document}
\footnotesize
% Einfache Singularitäten 



%%%%%%%%%%%%%%%%%%%%%%%%%%%%%

\begin{surferPage}
  \begin{surferTitle}La Viruta\end{surferTitle} \\
El ABC de las ecuaciones
\[8z^9-24x^2z^6-24y^2z^6+36z^8+24x^4z^3-168x^2y^2z^3\]
\[+24y^4z^3-72x^2z^5-72y^2z^5+54z^7-8x^6-24x^4y^2\]
\[-24x^2y^4-8y^6 + 36x^4z^2-252x^2y^2z^2+36y^4z^2\]
\[- 54x^2z^4-108y^2z^4 + 27z^6-108x^2y^2z + 54y^4z\]
\[-54y^2z^3 + 27y^4 = 0\]\\
\vspace{0.25cm}
¿Observaste con atención la ecuación de La Viruta? Parece bastante complicada.
Sin embargo, puede ser descrita en simples palabras: la parte superior tiene la forma de la letra griega $\alpha$ (Alfa), el borde derecho tiene la forma de una curva con un pico. Estas singularidades tienen el nombre de {\it pico}. Si arrastraramos ese pico a lo largo de la curva alfa, obtendríamos nuestra curva. Este tipo de superficies se las conoce como {\it producto cartesiano} en honor al matemático francés Ren\'e Descartes.\\
\vspace{0.25cm}
Los monomios de grado $1$ son $x$, $y$, $z$. Los monomios de grado $2$ son $x^2, xy, y^2, xz, yz, z^2$, etc. Cuanto mayor sea el grado, más monomios tenemos, lo que nos permite crear formas más complicadas. Es como un alfabeto: si tenemos más letras a nuestra disposición, podremos escribir palabras y frases mas complejas. 




  \begin{surferText}
     \end{surferText}
\end{surferPage}
%%%%%%%%%%%%%%%%%%%%%%%%%%%%%
\end{document}