\documentclass[es]{./../../common/SurferDesc}%%%%%%%%%%%%%%%%%%%%%%%%%%%%%%%%%%%%%%%%%%%%%%%%%%%%%%%%%%%%%%%%%%%%%%%
%
% The document starts here:
%
\begin{document}
\footnotesize
% Einfache Singularitäten 

%%% 1.Tafel
%%%%%%%%%%%%%%%%%%%%%%%%%%%%%

\begin{surferPage}
  \begin{surferTitle}Nozzle\end{surferTitle}   \\
Infinitely many letters in a word\\
\smallskip
\[y z (x^2	+ y - z)	= 0\]

\vspace{0.3cm}
Impressionists painted houses and meadows with thousands of coloured dots. Similarly, mathematical surfaces are formed by thousands of points, but points that have neither width nor mass but which solve the equation! \\
\vspace{0.3cm}
A way to imagine infinity is to start counting: $1, 2, 3,$ \dots\\
There is always a larger number and we will never manage to count to the end.\\
\vspace{0.3cm}
But not only the surface contains infinitely many points. Only between $0$ and $1$ there are infinitely many of them. This seems impossible? Just think that the points are infinitely small. They are painted with a pencil with zero thickness. You have to paint many of them to fill the line between $0$ and $1$, namely infinitely many.




  \begin{surferText}
     \end{surferText}
\end{surferPage}


\end{document}
%
% end of the document.
%
%%%%%%%%%%%%%%%%%%%%%%%%%%%%%%%%%%%%%%%%%%%%%%%%%%%%%%%%%%%%%%%%%%%%%%%
