\documentclass[es]{../../common/SurferDesc}%%%%%%%%%%%%%%%%%%%%%%%%%%%%%%%%%%%%%%%%%%%%%%%%%%%%%%%%%%%%%%%%%%%%%%%
%
% The document starts here:
%
\begin{document}
\footnotesize
% Einfache Singularitäten 

%%% 1.Tafel
%%%%%%%%%%%%%%%%%%%%%%%%%%%%%

\begin{surferPage}
  \begin{surferTitle}Cambalache \end{surferTitle}   \\
Infinitos puntos en una palabra\\
\smallskip
\[y z (x^2	+ y - z)	= 0\]

\vspace{0.3cm}
Del mismo modo que los impresionistas pintaban casas y prados con miles de puntos de pintura, las
superficies tambi\'en est\'an formadas por miles de puntos. De hecho, una infinidad, ¡todos soluciones de
una ecuaci\'on!\\
\vspace{0.3cm}
Una manera de pensar en el infinito es empezar a contar: uno, dos, tres, ... Siempre hay un n\'umero mayor, y no acabamos nunca de nombrar todos los n\'umeros naturales.\\
\vspace{0.3cm}
Fijate en la ecuaci\'on del Cambalache: el factor yz multiplica toda la ecuaci\'on. Por tanto todos los puntos
de los planos z=0 (horizontal) e y=0 (vertical) forman parte de \'el.
Pero no solamente figuras como el Cambalache, que contiene dos planos, est\'an formadas por una infinidad de
puntos. Un cuadrado, sin ir m\'as lejos, tambi\'en.\\
\vspace{0.3cm}
Parece imposible que infinitos puntos quepan en un cuadrado, que es una porci\'on finita, ¿verdad? Pens\'a
que los puntos son tan pequeños que se los considera sin dimensiones, y si pudi\'eramos dibujar realmente
uno, no ser\'ia perceptible al ojo humano.

  \begin{surferText}
     \end{surferText}
\end{surferPage}


\end{document}
%
% end of the document.
%
%%%%%%%%%%%%%%%%%%%%%%%%%%%%%%%%%%%%%%%%%%%%%%%%%%%%%%%%%%%%%%%%%%%%%%%
