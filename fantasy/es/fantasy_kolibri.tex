\documentclass[es]{../../common/SurferDesc}%%%%%%%%%%%%%%%%%%%%%%%%%%%%%%%%%%%%%%%%%%%%%%%%%%%%%%%%%%%%%%%%%%%%%%%
%
% The document starts here:
%
\begin{document}
\footnotesize
% Einfache Singularitäten 

%%%%%%%%%%%%%%%%%%%%%%%%%%%%%
\begin{surferPage}
  \begin{surferTitle}Colibrí\end{surferTitle}   \\
La ecuación determina los puntos\\
  
  \smallskip
\[z^3+ y^2	z^2	= x^2\]

\singlespacing
En términos algebraicos, el Colibrí esta compuesto por todos los puntos $(x, y, z)$ que cumplen la ecuación
\smallskip
\[ x^2= y^2z^2+z^3.\]
\smallskip
Por ejemplo, $(0,0,0),$ $(1,0,1)$ y $(3,-2,-3)$ son puntos pertenecientes al Colibrí, mientras que $(0,1,1)$ no lo es.\\
 \singlespacing
El mundo real en el que vivimos esta sujeto a tres dimensiones, arriba y abajo, izquierda y derecha, atrás y adelante. Estas direcciones las identificamos con los ejes $x$, $y$ y $z$. Cualquier punto del espacio se puede describir como un valor para cada varible. Estos valores se conocen como las coordenadas $(x,y,z)$ de ese punto.\\
\singlespacing
Si juntamos todos los puntos del espacio y coloreamos solo los que cumplen con la ecuación. Todos los puntos coloreados juntos forman esta imagen.\\
\singlespacing


  \begin{surferText}
     \end{surferText}
\end{surferPage}


\end{document}
%
% end of the document.
%
%%%%%%%%%%%%%%%%%%%%%%%%%%%%%%%%%%%%%%%%%%%%%%%%%%%%%%%%%%%%%%%%%%%%%%%
