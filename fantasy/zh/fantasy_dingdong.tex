\begin{surferPage}{叮咚}

通过改变方程来改变图象\\

\smallskip
\[x^2	+ y^2	+ z^3	= z^2\]

\singlespacing
叮咚的方程和形状都很简单。它的图象可由希腊字母$\alpha$绕它的对称轴旋转而得到。如果你倒过来看,叮咚就像一滴水。我们可以看到这滴水正在下落。
\newline
如果在这个方程中加入一个小参数$a$并让它连续的变化,我们可以创建一系列的图象,这些图象展示了这滴水出现的过程,它是如何接近它的结束位置并最终分离的。它就像一部电影的静止图像:
\smallskip

\[x^2	+ y^2	+ z^3	-z^2+0.1\cdot a=0.\]

\singlespacing
水滴在每一个时刻都处在引力和曲面张力相等的平衡状态。但这种平衡是不稳定的,它在落下之前会抖动。由数学家勒内·托姆发展的突变理论就是研究参数在怎样小的变化下会导致平衡状态的突然改变。
\end{surferPage}
