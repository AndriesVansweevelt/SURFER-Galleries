\documentclass[en]{./../../common/SurferDesc}%%%%%%%%%%%%%%%%%%%%%%%%%%%%%%%%%%%%%%%%%%%%%%%%%%%%%%%%%%%%%%%%%%%%%%%
%
% The document starts here:
%
\begin{document}
\footnotesize
% Einfache Singularitäten 


\begin{surferPage}
  \begin{surferTitle}Sweet\end{surferTitle}   \\

\smallskip
\[(x^2+ 9/4y^2	+ z^2- 1)^3- x^2z^3	- 9/80y^2z^3	= 0\]

\singlespacing
Love-letter
\singlespacing
It can`t go on and on like this!\\
This what we have we have to keep.\\
When we'll meet,\\
Something must be.\\
Together singularity.\\
What`s up on singularity?\\
For sure no platitudes.\\
Because what's singularity,\\
Is not wearing down so easily.\\
On a park bench in Grunewald\\
In two to face the rain,\\
Attempts in vain. Girl, take heart! Write me a card!\\
Yours, Bertie! (Erinner Singularity).\\
{\it Poem by Joachim Ringelnatz}
\singlespacing 
The passion of love is generally connected to the emotional power of a ''singularity''. This connection appears in many forms of art.
\singlespacing 
Try changing the last cube of the equation and replace it with a square.



  \begin{surferText}
     \end{surferText}
\end{surferPage}

\end{document}
%
% end of the document.
%
%%%%%%%%%%%%%%%%%%%%%%%%%%%%%%%%%%%%%%%%%%%%%%%%%%%%%%%%%%%%%%%%%%%%%%%
