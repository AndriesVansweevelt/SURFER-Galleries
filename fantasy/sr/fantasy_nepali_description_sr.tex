\documentclass[sr]{./../../common/SurferDesc}%%%%%%%%%%%%%%%%%%%%%%%%%%%%%%%%%%%%%%%%%%%%%%%%%%%%%%%%%%%%%%%%%%%%%%%
%
% The document starts here:
%
\begin{document}
\footnotesize
% Einfache Singularitäten 

\begin{surferPage}
  \begin{surferTitle}Непали\end{surferTitle}  \\
Свет без краја\\

\smallskip
\[(x y - z^3 -1)^2= (1 - x^2	- y^2)^3\]

\singlespacing
Можда вам се нека површ учини просто лепом и пожелите да је ставите у кристалну куглу пуну снега и да се њоме играте. Али немојте мислити да било коју површ можете да ставите у вашу дневну собу!
\\
\singlespacing
Постоје површи које се простиру у бесконачност и које, ма како биле лепе, никад не можете да ставите у кристалну куглу, ма које величине. У овом случају површ зовемо \textit{неограниченом}. Да бисмо обојили такву површ морамо да сакријемо неке њене делове.
\\
\singlespacing
Особина коју треба ограничити се не може лако препознати, чак ни уз помоћ SURFER-а. То је као када бисмо желели да сазнамо да ли је универзум ограничен: пошто су нам његове границе непознате, можда их има а можда и не.
  \begin{surferText}
     \end{surferText}
\end{surferPage}
%%%%%%%%%%%%%%%%%%%%%%%%%%%%%
%%%%%%%%%%%%%%%%%%%%%%%%%%%%%


\end{document}
%
% end of the document.
%
%%%%%%%%%%%%%%%%%%%%%%%%%%%%%%%%%%%%%%%%%%%%%%%%%%%%%%%%%%%%%%%%%%%%%%%
