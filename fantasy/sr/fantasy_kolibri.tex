\documentclass[sr]{./../../common/SurferDesc}%%%%%%%%%%%%%%%%%%%%%%%%%%%%%%%%%%%%%%%%%%%%%%%%%%%%%%%%%%%%%%%%%%%%%%%
%
% The document starts here:
%
\begin{document}
\footnotesize
% Einfache Singularitäten 

%%%%%%%%%%%%%%%%%%%%%%%%%%%%%
\begin{surferPage}
  \begin{surferTitle}Колибри\end{surferTitle}   \\
Једначина одређује тачке\\
  
  \smallskip
\[z^3+ y^2	z^2	= x^2\]

\singlespacing
Алгебарски речено, колибри се састоји од свих тачака  $(x, y, z)$ које задовољавају горњу једначину
\smallskip
\[ x^2= y^2z^2+z^3.\]
\smallskip
На пример, тачке, $(0,0,0),$ $(1,0,1)$ и $(3,-2,-3)$ припадају површи колибри, док $(0,1,1)$ не припада.\\
 \singlespacing
 Наш тродимензиони свет одређују три смера: напред-назад, лево-десно и горе-доле. Ови смерови се означавају са $x$, $y$ и $z$. Свака тачка простора се описује са по једном вредности у сваком смеру. Ове вредности су координате $(x,y,z)$ тачке.\\
\singlespacing
Све тачке простора са њиховим координатама  уносимо у једначину и бојимо само оне тачке за које је једначина задовољена. Све обојене тачке заједно стварају слику.\\
\singlespacing


  \begin{surferText}
     \end{surferText}
\end{surferPage}


\end{document}
%
% end of the document.
%
%%%%%%%%%%%%%%%%%%%%%%%%%%%%%%%%%%%%%%%%%%%%%%%%%%%%%%%%%%%%%%%%%%%%%%%
