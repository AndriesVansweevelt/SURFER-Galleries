\documentclass[en]{./../../common/SurferDesc}%%%%%%%%%%%%%%%%%%%%%%%%%%%%%%%%%%%%%%%%%%%%%%%%%%%%%%%%%%%%%%%%%%%%%%%
%
% The document starts here:
%
\begin{document}
\footnotesize
% Einfache Singularitäten 

%%% 1.Tafel
\begin{surferPage}
  \begin{surferTitle} Лимун\end{surferTitle}  \\ %%% Zitrus
Ово није лимун – варљивост слике\\
\smallskip
\[x^2 + z^2 = y^3 (1 - y)^3\] 


\singlespacing
Сигурно је да смо, када смо први пут бацили поглед на ову слику сви помислили:''То је лимун''. Али, ако је лимун, како то да нема ни мирис ни укус? Зашто нема мрље или поре? Јасно је да ово не може бити лимун! 
\singlespacing
Овај облик није лимун већ математички модел лимуна. Помаже нам да боље разумемо особине облика лимуна. У географији постоји одговарајући цитат: ''Мапа не представља територију.'' (Алфред Х. С. Корзибски) \\
\singlespacing

Једначинама правимо математичке моделе, који нам помажу да боље проучавамо облике ствари. 
\singlespacing
Све ово је део математичке поетике: помоћу алгебарских једначина можемо створити прелепе површи које одведу наше мисли у неочекиване делове нашег ума. 



  \begin{surferText}
     \end{surferText}
\end{surferPage}

\end{document}
%
% end of the document.
%
%%%%%%%%%%%%%%%%%%%%%%%%%%%%%%%%%%%%%%%%%%%%%%%%%%%%%%%%%%%%%%%%%%%%%%%
