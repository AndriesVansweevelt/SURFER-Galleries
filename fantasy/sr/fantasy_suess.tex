\documentclass[sr]{./../../common/SurferDesc}%%%%%%%%%%%%%%%%%%%%%%%%%%%%%%%%%%%%%%%%%%%%%%%%%%%%%%%%%%%%%%%%%%%%%%%
%
% The document starts here:
%
\begin{document}
\footnotesize
% Einfache Singularitäten 


\begin{surferPage}
  \begin{surferTitle}Слатко\end{surferTitle}   \\

\smallskip
\[(x^2+ 9/4y^2	+ z^2- 1)^3- x^2z^3	- 9/80y^2z^3	= 0\]

\singlespacing
Љубавно писмо
\singlespacing
Овако више не може!\\
Оно што имамо морамо задржати.\\
Када се будемо срели,\\
Нешто се мора десити.\\
Заједничка једнина.\\
Шта је то са једнином?\\
Сигурно то није тек фраза.\\
Јер оно што је једнина,\\
Не може се лако похабати.\\
На клупи у парку у Гринебалду\\
У пару пред кишом,\\
Узалудни покушаји. Девојко, узми срце! Напиши ми разгледницу!\\
Твој, Берти ! (Сети се једнине).\\
{\it Јоаким Рингелнац}
\singlespacing 
Љубавна страст се обично повезује са емоционалном моћи ''једнине''. Ова веза постоји у многим облицима уметности.
\singlespacing 
Покушајте да промените последњи кубни члан у једначини и да га замените квадратом.



  \begin{surferText}
     \end{surferText}
\end{surferPage}

\end{document}
%
% end of the document.
%
%%%%%%%%%%%%%%%%%%%%%%%%%%%%%%%%%%%%%%%%%%%%%%%%%%%%%%%%%%%%%%%%%%%%%%%
