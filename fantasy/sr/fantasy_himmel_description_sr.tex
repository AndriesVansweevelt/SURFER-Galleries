\documentclass[en]{./../../common/SurferDesc}%%%%%%%%%%%%%%%%%%%%%%%%%%%%%%%%%%%%%%%%%%%%%%%%%%%%%%%%%%%%%%%%%%%%%%%
%
% The document starts here:
%
\begin{document}
\footnotesize
% Einfache Singularitäten 

%%%%%%%%%%%

\begin{surferPage}
  \begin{surferTitle}Рај и Пакао\end{surferTitle}  \\
Правимо нове облике \\
\smallskip
\[x^2	- y^2z^2	= 0\]

\singlespacing
Да бисмо правили нове облике треба да разумемо како функционишу једначине. Њихови елементи се зову {\it мономи}, алгебарски изрази од слова и бројева.
\singlespacing
Моном може да садржи следеће елементе:
знак + или -, коефицијенте, променљиве, експоненте а има и свој степен.\\
\singlespacing
На пример: 
\smallskip
\[2xy^2z = +2x^1y^2z^1.\]
\\
\smallskip
Степен монома је збир свих експонената његових променљивих, у овом случају би био: $degree = 1+2+1=4$.  \\
\singlespacing
За прављење једначина се користе операције сабирања, одузимања и множења. Ове операције знамо из основне школе. Оне се користе за представљање свих алгебраских површи.
\singlespacing
Можете ли направити облике са испупчењима и наборима користећи само сабирање и множење?


  \begin{surferText}
     \end{surferText}
\end{surferPage}
\end{document}
%
% end of the document.
%
%%%%%%%%%%%%%%%%%%%%%%%%%%%%%%%%%%%%%%%%%%%%%%%%%%%%%%%%%%%%%%%%%%%%%%%
