\documentclass[sr]{./../../common/SurferDesc}%%%%%%%%%%%%%%%%%%%%%%%%%%%%%%%%%%%%%%%%%%%%%%%%%%%%%%%%%%%%%%%%%%%%%%%
%
% The document starts here:
%
\begin{document}
\footnotesize
% Einfache Singularitäten 




%%%%%%%%%%%%%%%%%%%%%%%%%%%%%

\begin{surferPage}
  \begin{surferTitle}Куаст\end{surferTitle} \\
Азбука једначина
  \smallskip
\[8z^9-24x^2z^6-24y^2z^6+36z^8+24x^4z^3-168x^2y^2z^3\]
\[+24y^4z^3-72x^2z^5-72y^2z^5+54z^7-8x^6-24x^4y^2\]
\[-24x^2y^4-8y^6 + 36x^4z^2-252x^2y^2z^2+36y^4z^2\]
\[- 54x^2z^4-108y^2z^4 + 27z^6-108x^2y^2z + 54y^4z\]
\[-54y^2z^3 + 27y^4 = 0\]\\
\vspace{0.3cm}
Да ли сте добро погледали једначину површи куаст? Изгледа веома компликовано. Сам облик се може описати једноставним речима: горња ивица је облика грчког слова $\alpha$, десна ивица има облик криве са испупчењем. Овакво испупчење се зове {\it шиљак}. Ако овај шиљак  повлачимо по кривој слова алфа добијамо куаст. Површи са овом особином се називају Декартови производи, у част француског математичара Рене Декарта.\\
\vspace{0.3cm}
Мономи првог степена су $x$, $y$, $z$. Мономи другог степена су $x^2, xy, y^2, xz, yz, z^2$ итд. Што је већи степен, постоји више таквих монома и ово нам омогућава прављење компликованијих површи. Слично је азбуци: више слова нам омогућава писање сложенијих речи и реченица. 




  \begin{surferText}
     \end{surferText}
\end{surferPage}
%%%%%%%%%%%%%%%%%%%%%%%%%%%%%
\end{document}