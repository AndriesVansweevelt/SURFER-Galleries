\begin{surferPage}{Mevlana'yla Şems}
Şapkadan yeni formlar çıkarmak... \\
\smallskip
\[x^2	- y^2z^2	= 0\]

\singlespacing
Yeni formlar yaratmak için denklemlerin nasıl çalıştığını anlamamız gerekir. Denklemlerin yapıtaşları {\it monomiallerdir}; bunlar harfler ve sayılardan oluşmuş cebirsel ifadelerdir.
\singlespacing
Bir monomial şu unsurları içerir:
İşaretler, katsayılar, değişkenler, üsler ve derece.\\
\singlespacing
Örneğin: 
\smallskip
\[2xy^2z = +2x^1y^2z^1.\]
\\
\smallskip
Bir monomialin  {\it derecesi}, değişkenlerinin üslerinin toplamıdır: yukarıdaki $derece = 1+2+1=4$.  \\
\singlespacing
Denklemler kurmak için toplama, çıkarma ve çarpma gibi aritmetik işlemleri kullanırız. Bu işlemleri ilkokuldan beri biliyoruz. Tüm cebirsel yüzeyler bunlar tarafından ifade edilir.
\singlespacing
Sadece toplama ve çarpma kullanarak sivriliklere ve deliklere sahip şekiller  yaratabilir misiniz?
\end{surferPage}
