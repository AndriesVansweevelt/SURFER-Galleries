\begin{surferPage}{Aşk}
\[(x^2+ 9/4y^2	+ z^2- 1)^3- x^2z^3	- 9/80y^2z^3	= 0\]

\singlespacing
Sibernetik
\singlespacing
üç kere üç dokuz eder\\
bilirsin\\
birin karesi birdir\\
kare kökü de\\
bilirsin\\
"mutlu aşk yoktur"\\
bilirsin\\
\vspace{0.4cm}

ama baharda ya da dışarda\\
sonsuz göğün altında\\
aşkın aşkla çarpımı\\
nedendir bilinmez\\
garip bir biçimde\\
hep sonsuzdur\\
kare kökü yoktur\\

{\it Turgut Uyar}
\singlespacing 
Aşk tutkusu çoğunlukla, sonsuzluk, tekillik gibi ifade edilemez şeylerin duygusal gücüyle ilişkilendirilir. Bu ilişki sanatın birçok biçiminde ortaya çıkar.
\singlespacing 
Eşitlikteki son küpü şairin peşinden giderek kareye dönüştürün bakalım ne olacak!
\end{surferPage}
