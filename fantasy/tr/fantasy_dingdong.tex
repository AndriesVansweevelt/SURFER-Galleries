\begin{surferPage}{Damla}
Formülü değiştirerek formu değiştirin...\\

\smallskip
\[x^2	+ y^2	+ z^3	= z^2\]

\singlespacing
Damla'nın formülü ve formu sade. Şekil,  Yunan harfi $\alpha$'nın kendi ekseni etrafında döndürülmesiyle elde edilmiş. Başaşağı bakarsanız Damla tam bir su damlası gibi görünür. Damlanın düşüşünü şöyle izleyebiliriz.
\newline

Denkleme küçük bir  $a$ parametresi ekler ve değerini sürekli biçimde değiştirirseniz bir dizi şekil elde edersiniz. Damla ortaya çıkar, yavaş yavaş son haline yaklaşır ve sonunda ikiye parçalanır.  Bir filmin kare kare görüntüleri gibi...

\[x^2	+ y^2	+ z^3	-z^2+0.1\cdot (a-0.5)=0.\]

Her bir anda damla yerçekimiyle eşitlenmiş yüzey gerilimi sayesinde bir denge konumundadır.
Fakat bu denge  kararlı olmadığından damla düşmeden önce  parçalara ayrılır. Matematikçi Ren\'e Thom tarafından çalışılmış afet (catastrophe) kuramı, parametrelerdeki küçük değişikliklerin dengede nasıl ani değişikliklere neden olduğunu inceler.
\end{surferPage}
