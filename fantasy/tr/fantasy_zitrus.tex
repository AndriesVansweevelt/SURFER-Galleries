\begin{surferPage}{Limon}
Bu bir limon değildir - imgelerin ihaneti...\\
\smallskip
\[x^2 + z^2 = y^3 (1 - y)^3\] 


\singlespacing
Bu resme ilk göz attığımızda şüphesiz hepimiz  ``Bu bir limon`` diye düşüneceğiz. Fakat bu bir limon ise niye kokusu ve tadı yok? Niye gözenekleri ve benekleri yok? Bu bir limon olamaz!
\singlespacing
Bu şekil bir limon değil elbette, onun matematiksel bir modeli. Limonun şeklinin özelliklerini daha iyi kavramamıza yardımcı oluyor. Coğrafyada  $Alfred\ H.\ S.\ Korzybski$ tarafından söylenmiş aynı havada bir cümle vardır: ''Bir harita bir ülke değildir.'' \\
\singlespacing

Formüller, formları daha iyi incelemek için bize yardımcı olacak matematiksel modeller kurmamızı sağlar.
\singlespacing
Bunları hepsi matematiğin şiirselliğinin parçaları:   düşüncelerimizi zihnimizin beklenmedik köşelerine taşıyan cebirsel denklemler aracılığıyla güzel yüzeyler yaratabiliyoruz.
\end{surferPage}
