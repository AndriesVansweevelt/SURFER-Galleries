\begin{surferPage}{Bocal}
Letras infinitas numa palavra\\
\smallskip
\[y z (x^2	+ y - z)	= 0\]

\vspace{0.3cm}
Os impressionistas pintaram casas e prados com milhares de pontos coloridos. De modo semelhante, as superf\'icies matem\'aticas s\~ao formadas por uma infinidade de pontos, pontos que n\~ao t\^em largura nem massa mas satisfazem a equa\c c\~ao!\\
\vspace{0.3cm}
Uma forma de imaginar o infinito \'e come\c car a contar: $1, 2, 3,$ \dots\\
H\'a sempre um n\'umero maior e n\'os nunca conseguimos chegar ao fim.\\
\vspace{0.3cm}
Mas n\~ao \'e s\'o a superf\'icie que cont\'em pontos em n\'umero infinito. S\'o entre $0$ e $1$ existe uma infinidade de pontos.  Parece imposs\'ivel? Basta pensar que os pontos s\~ao infinitamente pequenos. Poderiam ser pintados com um l\'apis de espessura zero. Ter\'iamos que pintar muitos pontos para preencher a linha entre $0$ e $1$. Na verdade,  uma infinidade...
\end{surferPage}
