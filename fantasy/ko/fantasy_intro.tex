\begin{surferIntroPage}{환상적인 곡면들}{fantasy_kolibri}{환상적인 곡면들}
많은 사람들은 수학을 복잡하고 어려운 것이라고 생각합니다. 그러나 사실 수학은 사물의 근본적인 구조와 공통적 성질을 파악할 수 있도록 하고, 세상을 이해하는 데 큰 도움을 줍니다. 
사물이 가진 중요한 공통점을 파악하고 중요하지 않은 성질을 무시함으로써, 사물을 여러가지 종류로 \textit{분류}할 수 있습니다. 

이러한 분류를 통해 우리는 무한한 다양성을 가진 사물에 대한 포괄적인 이해를 얻을 수 있습니다. 어떤 것이 중요하고 어떤 것이 중요하지 않은지 결정하는 것은 여러분이 무엇을 알고 싶은 지에 달려 있습니다. 크기와 사물의 구성 형태가 그 예입니다.
\\
\vspace{0.4cm}
형태를 묘사하고 분류하는 것은 인류의 오래된 숙제였고 그 해결책을 찾는 것 역시 어려운 문제였습니다. 고대 그리스인들은 주로 기하와 기하학적 비율을 이용하였습니다. 그 뒤 대수학는 아랍인(Al Khwarizmi, 900 B.C.)에 의해 발전되었습니다. 페르마와 데카르트는 기하학적 관계를 설명하기 위해 좌표계를 소개했으며, 이는 18세기 과학의 큰 성취로 여겨집니다. 좌표계는 대수와 기하를 접목시키는데 큰 역할을 하였습니다.
\\
\vspace{0.4cm}
SURFER라는 이 프로그램은 대수적 방정식으로부터 기하학적 이미지를 만들어 낸다는 측면에서 좌표계 활용의 가장 전형적인 예 입니다.  이 갤러리를 통해 여러분은 수학의 아름다움을 경험하고 더 창의적으로 발전할 수 있을 것 입니다. 오른쪽에서 마음에 듣는 곡면을 선택하십시오. 대수 곡면과 방정식의 형태의 관계가 쉽게 설명되어져 있습니다. \\
여러분의 상상력과 직관을 활용해 아름다운 곡면을 만들어보세요.
\end{surferIntroPage}
