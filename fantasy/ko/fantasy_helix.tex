\begin{surferPage}{나선}
비누막 보다 얇게\\
  \smallskip
\[6x^2	= 2x^4	+ y^2	z^2\]

\singlespacing
비누방울 외부 충격에 매우 민감합니다. 쳐다만 봐도 방울이 터질 것 같지 않나요? 비누방울의 표면은 물로 이루어진 안쪽 면과 비누로 이루어진 바깥쪽 면 두개의 면으로 이루어져 있습니다. 비누방울의 표면이 얇아지면, 예를들면  방울이 더 커질 때, 안쪽의 물이 방울을 터뜨려버립니다.\\
\vspace{0,3cm}
대수 곡면은 점들이 모인 면으로 이루어져 있기 때문에 비누방울보다도 더 얇습니다. 상상  속에서 질량이나 부피가 없는 면을 만들기 때문에 면이 뾰족점이나 주름을 갖더라도 터지지 않습니다.\\
\vspace{0,3cm}

예를 들어 여러분이 만약 실제 3D 나선을 만들고자 한다면 나선곡면보다는 훨씬 두껍게 만들어야 합니다.
\end{surferPage}
