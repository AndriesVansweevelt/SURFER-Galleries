\begin{surferPage}{Vis \`a Vis}
특이점 혹은 정점 - 친구 혹은 적\\
\smallskip
\[x^2	- x^3+ y^2+ y^4+ z^3- z^4	=  0\]

\vspace{0.3cm}
특이점들은 눈으로 쉽게 볼 수 있습니다. 왜냐하면 곡면이 매끄럽지 않은 부분, 예를 들면 뾰족한 끝이나 접힌 부분 이기 때문이죠. \\
\vspace{0.3cm}
Vis \`a Vis 곡면의 오른쪽에 있는 뾰족한 점은 특이점입니다. 반면 왼쪽에 있는 매끄러운 부분은 정칙점입니다. 특이점은 굉장히 흥미로운데 그 이유는 원래 식이 조금만 바뀌어도 특이점의 성질(위치, 개수 등)을  놀라울 정도로 많이 바뀌기 때문입니다. \\

\vspace{0.3cm}
어떤 사람들은 특이점을 연구하는데만 평생을 바치고 있다는 사실을 아시나요? 블랙홀이나 빅뱅도 우주론적 모델 방정식의 특이점들이랍니다. 그리고 여러분의 손가락 끝을 보십시오. 지문의 특이점들이 지문을 통해 사람을 식별할 수 있게 해줍니다!
\end{surferPage}
