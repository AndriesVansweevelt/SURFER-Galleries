\begin{surferPage}{딩동}
방정식을 변형하면서 곡면을 다른 모양으로 바꿔보세요!\\

\smallskip
\[x^2	+ y^2	+ z^3	= z^2\]

\singlespacing
딩동의 형태와 방정식은 상당히 간단합니다. 이 그림은 그리스 문자  $\alpha$ 를 회전시켜서 얻은 회천체와 같은 모양입니다. 딩동을 뒤집어서 보면 메달려있는 물방울을 볼 수 있습니다. 이제 여러분은 이 물방울이 떨어지는 모습을 보게 될 것입니다. 
\newline
작은 매개변수 $a$를 식에 더하고 오른쪽의 막대기를 이용하여 크기를 연속적으로 변화시키면 물방울이 떨어지면서 분리되는 모습을 볼 수 있습니다.  
\[x^2	+ y^2	+ z^3	-z^2+0.1\cdot a=0.\]

매 순간 물방울은 표면장력과 중력의 균형을 맞춥니다. 그러나 그 균형은 안정적이지 않기 때문에 물방울이 떨어지기 전에 크게 요동칩니다. 수학자 Ren\'e Thom의 재앙이론을 통해 매개변수의 작은 변화가 물리적 균형에 미치는 영향을 이해할 수 있습니다. 
\end{surferPage}
