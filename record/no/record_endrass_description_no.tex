\documentclass[no]{./../../common/SurferDesc}%%%%%%%%%%%%%%%%%%%%%%%%%%%%%%%%%%%%%%%%%%%%%%%%%%%%%%%%%%%%%%%%%%%%%%%
%
% The document starts here:
%
\begin{document}
\footnotesize
% Weltrekordfl�chen

%%% 1.Tafel

%%%%%%%%%%%%%%%%%%%%%%%%%%%%%
%%%%%%%%%%%%%%%%%%%%%%%%%%%%%%


\begin{surferPage}
  \begin{surferTitle}Endra�-flaten av �ttende grad\end{surferTitle} \\
	
	I 1995 konstruerte Stephan Endra� denne flaten av �ttende grad som hovedresultatet i hans 
	vitenskapelige avhandling ved Universitetet i Erlangen. Flaten har til sammen $168$ singulariteter, 
	som fortsatt er den gjeldende verdensrekorden. 
  	
	Gjennom et generelt resultat funnet av Varchenko, vet vi at et flater av �ttende 
	rad ikke kan ha mer enn $174$ singulariteter. Det vil si: $168 \le \mu(8) \le 174$. Det n�yaktige antallet er ukjent.
	
	Det var ikke lett � finne flaten. Endra� m�tte lete etter den i en $5$-dimensjonal familie med flater 
	av �ttende grad, hvor det gjennomsnittlige familiemedlemmet bare hadde $112$ singulariteter.

	I det interaktive bildet ser vi symmetrien til konstruksjonen: I tillegg til symmetrien til 
	et vanlig oktogon, er flaten symmetrisk med hensyn p� $xy$-planet.

	Uten � bruke slike symmetrier, ville s�kerommet ikke bare v�re av femte grad, men av enda h�yere dimensjoner. 
	
  \begin{surferText}
     \end{surferText}
\end{surferPage}
%%%%%%%%%%%%%%%%%%%%%%%%%%%%%




\end{document}
%
% end of the document.
%
%%%%%%%%%%%%%%%%%%%%%%%%%%%%%%%%%%%%%%%%%%%%%%%%%%%%%%%%%%%%%%%%%%%%%%%
