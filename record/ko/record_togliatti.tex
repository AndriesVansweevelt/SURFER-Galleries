\begin{surferPage}{Togliatti의 $5$차 방정식}
    Eugenio Giuseppe Togliatti는 $1937$년에 $31$개의 특이점을 갖는 $5$차식이 존재한다는 것을 증명하였습니다---이것은 그 당시 세계 기록이었습니다. 


    1980년에 Arneau Beauville는 코딩 이론을 이용하여 더 이상의 특이점이 존재하지 않는다는 것을 증명하였씁니다. 
    그 말인즉슨 Togliatti 의 기록은 절대 깨질 수 없다는 것이지요!

	정다면체의 대칭면을 통해 쿠머의 $4$차식이나 바스의 $5$차식과 유사한 $5$차 곡면을 만들 수 없기 때문에 $31$개의 특이점을 갖는 $5$차식은 평면 오각형의 대칭이라 부르는 더 적은 대칭면을 갖습니다.

여기서는 $1990$년에 울프 바스에 의해 발견된 식을 사용합니다.그 이유는 Togliatti 의 식은 시각화하기 쉽지 않기 때문입니다.
\end{surferPage}
