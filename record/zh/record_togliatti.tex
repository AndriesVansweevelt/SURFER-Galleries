\begin{surferPage}[陶里亚蒂5次曲面]{陶里亚蒂5次曲面}
1937年,欧亨尼奥朱塞佩陶里亚蒂证明了存在带有31个奇异点的5次曲面---当时的世界记录。
1980年,阿尔瑙伯维尔利用编码理论证明了5次曲面上的奇异点不可能更多。这意味着陶里亚蒂的结果是不可能被改进的。
不存在这样的柏拉图环体,它的对称平面可以被用来构造5次曲面,正如库默尔四次曲面或者巴斯六次曲面那样。 所以,带有31个奇异点的5次曲面具有较少的对称性,即只有平面五边形的对称性。
我们这里用到的方程是沃尔夫巴斯 (1990)发现的。没有采用陶里亚蒂当初的曲面,是因为他的曲面不容易被可视化
\end{surferPage}
