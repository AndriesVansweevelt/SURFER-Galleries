\documentclass[de]{./../../common/SurferDesc}%%%%%%%%%%%%%%%%%%%%%%%%%%%%%%%%%%%%%%%%%%%%%%%%%%%%%%%%%%%%%%%%%%%%%%%
%
% The document starts here:
%
\begin{document}
\footnotesize
% Weltrekordfl�chen

%%% 1.Tafel

%%%%%%%%%%%%%%%%%%%%%%%%%%%%%
\begin{surferPage}
  \begin{surferTitle}Eine Togliatti-Quintik\end{surferTitle}  \\

    Eugenio Giuseppe Togliatti bewies bereits im Jahr 1937,
    dass es eine Fl�che vom Grad $5$ (Quintik) mit genau $31$ Singularit�ten
    gibt --- damals Weltrekord!  

    1980 gelang es Arneau Beauville durch eine interessante Beziehung
    zur Kodierungstheorie zu zeigen, dass eine Quintik nicht
    mehr Singularit�ten besitzen kann. 
    Dies hei�t also, dass Togliattis Weltrekord niemals 
    mehr verbessert werden kann!

    Da es leider keinen Platonischen K�rper gibt, dessen Symmetrieebenen man
    ausnutzen k�nnte, um eine Fl�che vom Grad $5$ in Anlehnung an Kummers
    Quartik und Barths Sextik zu konstruieren, besitzt die abgebildete Quintik
    mit $31$ Singularit�ten weniger Symmetrie, n�mlich die Symmetrie eines
    ebenen F�nfecks.

    Auch diese Gleichung hat Wolf Barth gefunden (1990); Togliattis Fl�che 
    von 1937 ist n�mlich schwer zu visualisieren.



  \begin{surferText}
     \end{surferText}
\end{surferPage}


\end{document}
%
% end of the document.
%
%%%%%%%%%%%%%%%%%%%%%%%%%%%%%%%%%%%%%%%%%%%%%%%%%%%%%%%%%%%%%%%%%%%%%%%
