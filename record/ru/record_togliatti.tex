\documentclass[ru]{./../../common/SurferDesc}%%%%%%%%%%%%%%%%%%%%%%%%%%%%%%%%%%%%%%%%%%%%%%%%%%%%%%%%%%%%%%%%%%%%%%%
%
% The document starts here:
%
\begin{document}
\footnotesize
% WeltrekordflŠchen

%%% 1.Tafel

%%%%%%%%%%%%%%%%%%%%%%%%%%%%%
\begin{surferPage}
  \begin{surferTitle}Квинтика Тольятти\end{surferTitle}  \\

Эухенио Джузеппе Тольятти доказал уже в 1937 г., что существует поверхность пятого порядка (квинтика) ровно с $31$ сингулярностью – тогдашний мировой рекорд!


В 1980 г. Арно Бовиллю удалось показать благодаря интересной взаимосвязи с теорией кодирования, что квинтика не может иметь ещё больше сингулярностей. Т.е., «мировой рекорд» Тольятти улучшить уже нельзя.

Т.к. нет платоновых тел, плоскости симметрии которых можно было бы использовать для того, чтобы сконструировать поверхность пятого порядка, опираясь на квартику Куммера (поверхность 4-го порядка) и секстику Барта (поверхность шестого порядка), то изображённая квинтика с $31$ сингулярностью обладает меньшей симметрией, а, точнее говоря, симметрией плоского пятиугольника.

Соответствующее уравнение составил также Вольф Барт в 1990 г. Поверхность Тольятти 1937 года сложно поддаётся визуализации.
  \begin{surferText}
     \end{surferText}
\end{surferPage}


\end{document}
%
% end of the document.
%
%%%%%%%%%%%%%%%%%%%%%%%%%%%%%%%%%%%%%%%%%%%%%%%%%%%%%%%%%%%%%%%%%%%%%%%
