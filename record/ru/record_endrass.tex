\documentclass[ru]{./../../common/SurferDesc}%%%%%%%%%%%%%%%%%%%%%%%%%%%%%%%%%%%%%%%%%%%%%%%%%%%%%%%%%%%%%%%%%%%%%%%
%
% The document starts here:
%
\begin{document}
\footnotesize
% WeltrekordflŠchen

%%% 1.Tafel

%%%%%%%%%%%%%%%%%%%%%%%%%%%%%
%%%%%%%%%%%%%%%%%%%%%%%%%%%%%%


\begin{surferPage}
  \begin{surferTitle}Октика Эндрасса\end{surferTitle} \\
Штефан Эндрасс сконструировал эту поверхность 8-й степени (поэтому и название – октика) в 1995 г. при подготовке своей диссертации в Эрлангене. Эта поверхность имеет в общей сложности $168$ сингулярностей: мировой рекорд по сей день!
  
     Опираясь на общий результат, полученный Варченко, известно, что октика может иметь не более $174$ сингулярностей, т.е.: $168 \le \mu(8) \le 174$. 
    Точное число ещё не известно.

     Найти поверхность было непросто; Эндрассу нужно было искать её в пятимерном семействе октик, имеющих в общем случае лишь $112$ сингулярностей.

На изображении достаточно хорошо видна симметрия октики: дополнительно к симметрии правильного восьмиугольника поверхность зеркально симметрична относительно плоскости $xy$.

Без использования такой сложной симметрии искомое пространство было бы не пятимерным, а существенной большей размерности!
  \begin{surferText}
     \end{surferText}
\end{surferPage}
%%%%%%%%%%%%%%%%%%%%%%%%%%%%%




\end{document}
%
% end of the document.
%
%%%%%%%%%%%%%%%%%%%%%%%%%%%%%%%%%%%%%%%%%%%%%%%%%%%%%%%%%%%%%%%%%%%%%%%
